This chapter provides a quick overview to using Nek5000
for some basic flow problems provided in the {\tt .../examples}
directory.
           
Nek5000 runs under Linux or any Linux-like OS such as MAC, AIX, BG, Cray etc.  
The source is maintained in an svn repository and can be downloaded
from the Nek5000 homepage (google nek5000) or, on linux systems,
with the svn checkout command:

\begin{verbatim}
svn co https://svn.mcs.anl.gov/repos/nek5 nek5_svn
\end{verbatim}
\noindent
After downloading, build the tools by typing
\begin{verbatim}
cd nek5_svn/trunk/tools
maketools all
\end{verbatim}
\noindent
which will put the tools {\tt genbox, genmap, n2to3, postx,
prex,} and {\tt pretex} in the top-level {\tt /bin} directory
(and will create {\tt /bin} if it does not exist).
It may be necessary to edit the {\tt maketools} file to change 
the compilers (e.g., to pgf77/pgcc or ifort/icc).  However,
the default gfortran/gcc is generally fine.
%\footnote{In some cases it may be necessary to reduce the memory
%footprint of some of the tools. The procedures are (will be)
%described in \sc{sec. troubleshooting}.}

In addition to the compiled tools, there are numerous scripts in
\begin{verbatim}
nek5_svn/trunk/tools/scripts
\end{verbatim}
that are useful to have in the execution path, achieved 
either by adding this directory to the path or copying its
contents to the top-level /bin directory.  In the following,
we assume that scripts such as {\tt nek} and {\tt nekb}
are in the path.  We further assume that the {\tt nek5\_svn/}
prefix is implied in any future directory reference, unless
otherwise specified.

