\section{Solving a Convection-Diffusion problem in Nek5000}


Consider a SEM discretized higher order passive scalar given by an advection-diffusion equation. Following Eq.~\ref{eq:advb} and assuming that any varaible coefficients are already account for insside the discrete operators we have
\begin{equation}\label{eq:passscal}
M \frac{d u}{dt} = \, Au -Cu + M f, 
\end{equation}

The equation is solved using an Implcit/Explicit scheme (IMEX), in this case BDF for the implicit part and EXT for the explicit step, leading to

\begin{equation}
\sum\limits_{j=0}^k \frac{b_j}{\Delta t} M\vect u^{n+1-j}  = Au^{n+1} -Cu^{n+1} + M f^{n+1}
\end{equation}

Which by extrapolation the convective term $Cu$ and regrouping the terms leads to
\begin{equation}
 \frac{b_0}{\Delta t} M\vect u^{n+1} -Au^{n+1} = -\sum\limits_{j=1}^k a_j C\vect u^{n+1-j} -\sum\limits_{j=1}^k \frac{b_j}{\Delta t} M\vect u^{n+1-j}+ M f^{n+1}
\end{equation}


This leads to a simple Helmholtz system to solve
\begin{equation}
 (\frac{b_0}{\Delta t} M -A)u^{n+1} = -\sum\limits_{j=1}^k a_j C\vect u^{n+1-j} -\sum\limits_{j=1}^k \frac{b_j}{\Delta t} M\vect u^{n+1-j}+ M f^{n+1}
\end{equation}

In Nek5000 the convention for the forcing terms for passive scalars is to use the letter $q$. We first build the right hand side in {\tt makeq} as follows
\begin{itemize}
\item {\tt makebdq} sums the last two terms of the RHS (the lagged BDF terms and the forcing)
\item {\tt makeabq} add to the RHS the extrapolated convection term thus producing the final right hand side
\end{itemize}

Next the Helmholtz problem is setup in {\tt conduct.f}, more specifically in the routine {\tt cdscal}. To setup the Helmholtz problem the routine {\tt sethelm} is called and the coeffcients are given as $H_1u+H_2u$, in this case $H_2=\frac{b_0}{\Delta t} M $ and $H_1=-A$.

Last step is to solve the Helmholtz which is performed using {\tt hsolve}.