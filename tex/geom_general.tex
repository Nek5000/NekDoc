The domain in which the fluid flow/heat transfer
problem is solved consists of two distinct subdomains. The
first subdomain is that part of the region occupied by
fluid, denoted \(\Omega_f\), while the second subdomain is that part
of the region occupied by a solid, denoted \(\Omega_s\). These two
subdomains are depicted in Fig.~\ref{fig:domains}. The entire domain is denoted as \(D=\Omega_f \cup \Omega_s\).
The fluid problem is solved in the domain \(\Omega_f\), while the
temperature in the energy equation is solved in the
entire domain; the passive scalars can be solved in either
the fluid or the entire domain.
  
We denote the entire boundary of \(\Omega_f\) as \(\partial \Omega_f\), that part
of the boundary of \(\Omega_f\) which is not shared by \(\Omega_s\) as
\(\overline{\partial \Omega_f}\), and
that part of the boundary of \(\Omega_f\) which is shared by \(\Omega_s\).
In addition, \(\partial \Omega_{s}, \overline{\partial \Omega_s}\) are analogously defined.
These distinct portions of the
domain boundary are illustrated in Fig.\ref{fig:domains}.
The restrictions on the domain for Nek5000 are itemized below.
\begin{itemize}
\item The domain \(\Omega=\Omega_f \cup \Omega_s\) must correspond either to a
  planar (Cartesian) two-dimensional geometry, or to the
  cross-section of an axisymmetric region specified by
  revolution of the cross-section about a specified axis, or
  by a (Cartesian) three-dimensional geometry.
\item For two-dimensional and axisymmetric geometries, the
  boundaries of both subdomains, \(\partial \Omega_f\) and
  \(\partial \Omega_s\), must be
  representable as (or at least approximated by) the union of
  straight line segments, splines, or circular arcs.
\item Nek5000 can interpret a two-dimensional image as either
  a planar Cartesian geometry, or
  the cross-section of an axisymmetric body. In the case of
  the latter, it is assumed that the y-direction is the radial
  direction, that is, the axis of revolution is at y=0.
  Although an axisymmetric geometry is, in fact,
  three-dimensional, Nek5000 can assume that the field variables
  are also axisymmetric ( that is, do not depend on azimuth,
  but only \(y\), that is, radius, \(x\), and \(t\) ), thus reducing the
  relevant equations to "two-dimensional" form.
\end{itemize}

Fully general three-dimensional meshes generated by other softwares
packages can be input to PRENEK as import meshes.
%\subsection{Description and implementation}  
\section{Moving Geometry}
If the imposed boundary conditions allow for motion
of the boundary during the solution period (for example,
moving walls, free-surfaces, melting fronts, fluid layers),
then the geometry of the computational domain is automatically
considered in Nek5000 as being time-dependent.

For time-dependent geometry problems,
a mesh velocity {\bf w} is defined at each
collocation point of the computational domain (mesh) to
characterize the deformation of the mesh.
In the solution of the mesh velocity, the value of the mesh
velocity at the moving boundaries is first computed
using appropriate kinematic conditions (for free-surfaces, moving walls
and fluid layers) or dynamic conditions (for melting fronts).
On all other external boundaries, the normal mesh velocity on the
boundary is always set to zero.
In the tangential direction, either a zero tangential velocity
condition or a zero tangential traction condition is imposed; this
selection is automatically performed by Nek5000 based on
the fluid and/or thermal boundary conditions specified
on the boundary.
However, under special circumstances the user may want
to override the defaults set by Nek5000, this is
described in the PRENEK manual in Section 5.7.\footnote{This manual is old may soon be deprecated}
If the zero tangential mesh velocity is imposed, then the mesh
is fixed in space; if the zero traction condition is imposed,
then the mesh can slide along the tangential directions on
the boundary.
The resulting boundary-value-problem for the mesh velocity is solved
in Nek5000 using a elastostatic solver, with the Poisson ratio
typically set to zero.
The new mesh geometry is then computed by integrating the
mesh velocity explicitly in time and updating the nodal coordinates of the
collocation points.

Note that the number of macro-elements, the order of the macro-elements
and the topology of the mesh remain {\em unchanged} even though
the geometry is time-dependent.
The use of an arbitrary-Lagrangian-Eulerian description in Nek5000
ensures that the moving fronts are tracked with the minimum amount
of mesh distortion;
in addition, the elastostatic mesh solver can handle moderately
large mesh distortion.
However, it is the responsibility of the user to decide when
a mesh would become "too deformed" and thus requires remeshing.
The execution of the program will terminate when the mesh becomes
unacceptable, that is, a one-to-one mapping between the physical
coordinates and the isoparametric local coordinates for any
macro-element no longer exists.