\documentclass[11pt]{report}              % Book class in 11 points
\ifdefined\HCode
\usepackage{tex4ht}
\else
\usepackage{mcs_pack}
\fi
\usepackage{chngcntr}

\input{pf.tex}
\input{psfig.tex}
\input{commands.tex}
\usepackage{fancyvrb}
\newcommand\verbbf[1]{\textcolor[rgb]{0,0,1}{#1}}
\usepackage{tabularx}  % for 'tabularx' environment and 'X' column type
\usepackage{ragged2e}  % for '\RaggedRight' macro (allows hyphenation)
\newcolumntype{Y}{>{\RaggedRight\arraybackslash}X} 
\usepackage{booktabs}  % for \toprule, \midrule, and \bottomrule macros\documentclass[10pt]{•}
%\parindent0pt  
%\parskip10pt             % make block paragraphs
%\raggedright                            % do not right justify

\title{Nek5000 User Documentation}    % Supply information
\author{Paul Fischer, James Lottes,  Stefan Kerkemeier, Oana Marin, Katherine Heisey, Aleks
Obabko, Elia Merzari, Yulia Peet}              %   for the title page.
\ifdefined\HCode\else
\publicationdate{date set on next release}
\report_no{ANL/MCS-TM-351}
\fi
\date{\today}                           %   Use current date. 
%\makeindex
% Note that book class by default is formatted to be printed back-to-back.
\begin{document}                        % End of preamble, start of text
%\frontmatter                            % only in book class (roman page #s)
%\mainmatter      

\maketitle
\clearpage
\thispagestyle{empty}
\pagenumbering{roman}
\tableofcontents
\clearpage

\section*{Executive Summary} 


Abstract here.

\addcontentsline{toc}{section}{Executive Summary}
\newpage

\pagenumbering{arabic}
\setcounter{page}{1}

\chapter{Introduction}

Nek5000 is designed to simulate laminar, transitional, and turbulent
incompressible or low Mach-number flows with heat transfer and species
transport. It is also suitable for incompressible magnetohydrodynamics
(MHD). Nek5000 is written in f77 and C. It uses MPI for message passing
(but can be compiled without MPI for serial applications)
and some LAPACK routines for eigenvalue computations (depending on
the particular solver employed).  In addition, it can be optionally
coupled with MOAB, which provides an interface to meshes generated with 
CUBIT. Nek5000 output formats can be read by either {\tt postx} or the parallel visualization 
package VisIt developed by Hank Childs and colleagues at LLNL/LBNL.
VisIt is mandatory for large problems (e.g., more than 100,000 
spectral elements).
\subsection*{Computational approach}
The spatial discretization is based on the spectral element method (SEM) \cite{pat84}, which is a high-order weighted residual technique similar to the finite element method.   
In the SEM, the solution and data are represented in terms of 
\(N\)th-order tensor-product polynomials within each of \(E\) deformable 
hexahedral (brick) elements. Typical discretizations involve \(E\)=100--10,000 elements of order 
\(N\)=8--16 (corresponding to 512--4096 points per element).
Vectorization and cache efficiency derive from the local lexicographical
ordering within each macro-element and from the fact that the action of
discrete operators, which nominally have \(O(EN^6)\) nonzeros, can be evaluated
in only \(O(EN^4)\) work and \(O(EN^3)\) storage through the use of 
tensor-product-sum factorization \cite{sao80}.   The SEM exhibits 
very little numerical dispersion and dissipation, which can be important, 
for example, in stability calculations, for long time integrations, 
and for high Reynolds number flows. We refer to \cite{dfm02} for more
details.
\input{philosophy}
Nek5000 solves the unsteady incompressible two-dimensional,
axisymmetric, or three-dimensional Stokes or Navier-Stokes
equations with forced or natural convection heat transfer in
both stationary (fixed) or time-dependent geometry. It also solves the compressible Navier-Stokes in the Low Mach regime, the magnetohydrodynamic equation (MHD).
The solution variables are the fluid velocity
\(\bu=(u_{x},u_{y},u_{z})\), the pressure \(p\),
the temperature \(T\). 
%, mesh velocity \({\bf w}=(w_{x},w_{y},w_{z})\) (for time-dependent geometry).
%,independent passive scalar fields \(\phi_{i}\), {\footnotesize i=1,2,\ldots},
%and magnetic field \({\bf B}=(B_{x},B_{y},B_{z})\) (for MHD).
All of the above field variables
are functions of space \({\bf x}=(x,y,z)\) and time \(t\)
in domains \(\Omega_f\) and/or \(\Omega_s\) defined in Fig. \ref{fig:domains}.
Additionally Nek5000 can handle conjugate heat transfer problems.

\begin{figure}
\centering
%\includegraphics[width=0.3\textwidth]{Figs/domains}
\includegraphics[width=0.6\textwidth]{Figs/walls}
\caption{Computational domain showing respective fluid
and solid subdomains, \(\Omega_f\) and \(\Omega_s\). The shared boundaries are denoted \(\partial\Omega_f=\partial\Omega_s\) and the solid boundary which is not shared by fluid is \(\overline{\partial\Omega_s}\), while the fluid boundary not shared by solid \(\overline{\partial\Omega_f}\).}
\label{fig:domains}
\end{figure}

\subsubsection{Incompressible Navier--Stokes equations}
%
The governing equations of flow motion in dimensional form are
\begin{eqnarray}\label{eq:ns_momentum}
\rho(\partial_{t} \bu +\bu \cdot \nabla \bu) = - \nabla p + \nabla \cdot \tau + \rho {\bf f} \,\, , \text{in } \Omega_f , \quad \text{  (Momentum)  } 
\end{eqnarray}
where \( \tau=\mu[\nabla \vect u+\nabla \vect u^{T}]\).
\begin{eqnarray}\label{eq:ns_cont}
 \nabla \cdot \bu =0 \,\, , \text{in } \Omega_f, \quad \text{  (Continuity)  }   
\end{eqnarray}

If the fluid viscosity is constant in the entire domain the viscous stress tensor can be contracted \(\nabla\cdot\tau=\mu\Delta \vect u\), therefore one may solve the Navier--Stokes equations in either the stress formulation, or no stress

\begin{itemize}
\item Variable viscosity requires the full stress tensor \(\nabla \cdot \tau=\nabla \cdot \mu[\nabla \vect u+\nabla \vect u^{T}]\), and we shall refer to this as the stress formulation
\item Constant viscosity leads to a simpler stress tensor \(\nabla \cdot \tau=\mu\Delta \vect u\), which we refer to as the 'no stress' formulation
\end{itemize}

\subsubsection{Non-dimensional Navier-Stokes}
Let us introduce the following non-dimensional variables \(\vect x^*\ = \frac{\vect x}{L}\), \(\vect u^*\ = \frac{u}{U}\), \(t^*\ = \frac{t}{L/U}\,\).
For the pressure scale we have two options 
\begin{itemize}
\item convective effects are dominant i.e. high velocity flows
\( p^* = \frac{p}{\rho U^2} \)
\item viscous effects are dominant i.e. creeping flows (Stokes flow)
\( p^* = \frac{p L}{\mu U} \)
\end{itemize}
For highly convective flows we choose the first scaling of the pressure and obtain the non-dimensional Navier-Stokes:
\begin{equation}\label{eq:NS_nondim}
\frac{\partial \mathbf{u^*}}{\partial t^*} + \mathbf{u^*} \cdot \nabla \mathbf{u^*}\ = -\nabla p^* + \frac{1}{Re} \nabla\cdot \tau^* + \frac{1}{Fr}\frac{\mathbf{f}}{g}.
\end{equation}
where \( \tau^*=[\nabla \vect u^*+\nabla \vect u^{*T}]\).
The two non-dimensional numbers here are the Reynolds number \(Re=\frac{\nu}{U L}\) \(Fr\) and the Froude number, defined as \(Fr = \frac{U^2}{gL}\).
%The above governing equations are subject to boundary conditions
%and initial conditions described in the following sections.
%
%\begin{tabular}{ l|l|l|l| }
%   \hline
%   Equation &  & &\\ \hline \hline
%   Incompressible NS (stress)& \ref{eq:ns_momentum} & \ref{eq:ns_cont}, \((4)=0\) & \\ 
%   Incompressible NS (non-stress)& \ref{eq:ns_momentum}, \((3)=\Delta \vect u\) & \ref{eq:ns_cont}, \((4)=0\) \\ 
%   Incompressible NS (non-stress)+Energy& \ref{eq:ns_momentum}, \((3)=\Delta \vect u\) & \ref{eq:ns_cont},\((4)=0\)& \ref{eq:energy} \\ 
%   Unsteady Stokes (non-stress)& \ref{eq:ns_momentum}, \((2)=0\( & \ref{eq:ns_cont},\((4)=0\)&  \\ 
%   Steady Stokes (non-stress)& \ref{eq:ns_momentum}, \((1)=0\), \((2)=0\) & \ref{eq:ns_cont},\((4)=0\)&  \\
%   Steady Heat transfer& --- & ---&  \ref{eq:energy}, \((5)=0\), \((6)=0\)\\
%   Compressible NS, Low Mach (non-stress)& \ref{eq:ns_momentum}, \((3)=\Delta \vect u\) & \ref{eq:ns_cont}, \((4)\neq 0\) \ref{eq:energy} + EOS &\ref{eq:energy} \\ 
%   \hline
%\end{tabular}

%\newline
\subsubsection{Energy equation}
In addition to the fluid flow, Nek5000 computes automatically the energy equation
\begin{eqnarray}\label{eq:energy}
 \rho c_{p} ( \partial_{t} T + \bu \cdot \nabla T ) =
   \nabla \cdot (k \nabla T) + q_{vol}\,\, ,\text{in } \Omega_f\cup \Omega_s  \text{  (Energy)  } 
\end{eqnarray}

\subsubsection{Non-dimensional energy/passive scalar equation}
A similar non-dimensionalization as for the flow equations using the non-dimensional variables
\(\vect x^*\ = \frac{\vect x}{L}\),  \(\vect u^*\ = \frac{u}{U}\), \(t^*\ = \frac{t}{L/U}\), \(T=\frac{T^*-T_0}{\delta T}\) leads to
\begin{eqnarray}\label{eq:energy_nondim}
\partial_{t^*} T^* + \vect u^* \cdot \nabla T^* =
  \frac{1}{Pe} \nabla \cdot \nabla T^* + q_{vol}\,\, ,\text{in } \Omega_f\cup \Omega_s  \text{  (Energy)  } 
\end{eqnarray}
where \(Pe=LU/\alpha\), with \(\alpha=k/\rho c_p\).


\subsubsection{Passive scalars}\label{sec:passive_scal}

We can additionally solve a convection-diffusion equation for each passive scalar \(\phi_i\),
\(i\)=1,2,\(\ldots\) in \(\Omega_f \cup \Omega_s\)
\begin{eqnarray}\label{eq:pass_scal}
   (\rho c_{p})_i ( \partial_{t} \phi_{i} + \bu \cdot \nabla \phi_{i} ) =
   \nabla \cdot (k_i \nabla \phi_{i}) + (q_{vol})_i.
\end{eqnarray}

The terminology and
restrictions of the temperature equations are retained for
the passive scalars, so that it is the responsibility of the
user to convert the notation of the passive scalar
parameters to their thermal analogues.
For example, in the context of mass transfer,
the user should recognize that the values specified
for temperature and heat flux
will represent concentration and mass flux, respectively.
Any combination of these equation characteristics is permissible with the
following restrictions. First, the equation must be set to unsteady if it is
time-dependent or if there is any type of advection. For these cases, the
steady-state (if it exists) is found as stable evolution of the
initial-value-problem. Secondly, the stress formulation must be selected if
the geometry is time-dependent. In addition, stress formulation must be
employed if there are traction boundary conditions applied on any fluid
boundary, or if any mixed velocity/traction boundaries, such as symmetry and
outflow/n, are not aligned with either one of the Cartesian \(x,y\) or \(z\) axes.
Other capabilities of Nek5000 are the linearized Navier-Stokes for flow stability, magnetohydrodynamic flows etc.



\subsubsection{Unsteady Stokes }
In the case of flows dominated by viscous effects Nek5000 can solve the reduced Stokes equations
\begin{eqnarray}\label{eq:ns_momentum_stokes}
 \rho(\partial_{t} \bu ) = - \nabla p + \nabla \cdot \tau + \rho {\bf f} \,\, , \text{in } \Omega_f \text{  (Momentum)  }
\end{eqnarray}
where \(\nabla \cdot\tau=\nabla\cdot\mu[\nabla \vect u+\nabla \vect u^{T}]\) and
\begin{eqnarray}\label{eq:ns_cont_stokes}
 \nabla \cdot \bu =0 \,\, , \text{in } \Omega_f  \text{  (Continuity)  } 
\end{eqnarray}
Also here we can distinguish between the stress and non-stress formulation according to whether the viscosity is variable or not. The non-dimensional form of these equations can be obtained using the viscous scaling of the pressure.


\subsubsection{Steady Stokes }
If there is no time-dependence, then Nek5000 can further reduce to
\begin{eqnarray}\label{eq:ns_momentum_steady_stokes}
 - \nabla p + \nabla \cdot \tau + \rho {\bf f}=0 \,\, , \text{in } \Omega_f \text{  (Momentum)  }
\end{eqnarray}
where \(\nabla \cdot\tau=\nabla\cdot\mu[\nabla \vect u+\nabla {\vect u}^{T}]\) and
\begin{eqnarray}\label{eq:ns_cont_steady_stokes}
 \nabla \cdot \bu =0 \,\, , \text{in } \Omega_f  \text{  (Continuity)  } 
\end{eqnarray}

\subsection{Linearized Equations}
In addition to the basic evolution equations described above, Nek5000
provides support for the evolution of small perturbations about
a base state by solving the {\em linearized equations}
\begin{equation} \label{eq:pertu}
  \rho(\partial_{t} {\bu_i}' + \bu \cdot \nabla {\bu_i}^{'} + \bu_i' \cdot \nabla \bu) =
   - \nabla p_i' + \mu \nabla^2 \bu_i', \qquad \nabla \cdot \bu_i' = 0,
\end{equation}
for multiple perturbation fields \(i=1,2,\dots\) subject to different initial
conditions and (typically) homogeneous boundary conditions.  
These solutions can be evolved concurrently with the base fields \((\bu,p,T)\).
There is also
support for computing perturbation solutions to the Boussinesq equations for
natural convection.  Calculations such as these can be used to estimate Lyapunov exponents of chaotic flows, etc.



\subsection{Steady conduction}    
The energy Eq.~\ref{eq:energy} in which the advection term \(\bu \cdot \nabla T\)
    and the transient term \(\partial_{t} T\) are zero. In essence this represents a Poisson equation.
    

\subsection{Low-Mach Navier-Stokes}\label{sec:lowma}
The compressible Navier-Stokes differ mathematically from the incompressible ones mainly in the divergence constraint \(\nabla \cdot \vect u\neq 0\). In this case the system of equations is not closed and an additional equation of state (EOS) is required to connect the state variables, e.g. \(p=f(\rho,T)\). However Nek5000 can only solve the Low Mach approximation of the compressible Navier-Stokes. The Low-Mach approximation decouples the pressure from the velocity leading to a system of equations which can be solved numerically in a similar fashion as the incompressible Navier-Stokes.

The Low Mach equations in non-dimensional form are 
\begin{eqnarray}
&&\rho\bigg(\frac{\d \vect u}{\d t}+ \vect u\cdot\nabla\vect u\bigg)=-\nabla p+\nabla \cdot\bb\tau+\rho\vect f\ \\ \nonumber
&&\rho\bigg(\frac{\d \vect \rho}{\d t}+ \vect u\cdot\nabla\vect \rho\bigg)=-\nabla \cdot \vect u\\ \nonumber
&&\rho\bigg(\frac{\d T}{\d t}+ \vect u\cdot\nabla T\bigg)=-\nabla \cdot k \nabla T\\ \nonumber
\end{eqnarray}
where \(\tau=\mu[\nabla \vect u+\nabla \vect u^{T}-\frac{2}{3}\nabla \cdot \vect u \vect I]\).


The implementation of the equation if state for the Low Mach formulation is for the moment hard-coded to be the ideal gas equation of state \(p=\rho R T\). This allows for both variable density and variable viscosity. The system is solved by substituting \(\rho=f(p,T)\) into the continuity equation and obtaining a so-called thermal divergence (the term \(\nabla \cdot u\) is given as a function of the temperature).
A more detailed description on how these equations connect is given in section \ref{sec:lowma} as well as in the developer's manual.

\subsection{Incompressible MHD equations}\label{sec:mhd}
Magnetohydrodynamics is based on the idea that magnetic fields can induce currents in a moving conductive fluid, which in turn creates forces on the fluid and changing the magnetic field itself. The set of equations which describe MHD are a combination of the Navier-Stokes equations of fluid dynamics and Maxwell's equations of electromagnetism. These differential equations have to be solved simultaneously, and Nek5000 has an implementation for the incompressible MHD.

Consider a fluid of velocity \(\vect u\) subject to a magnetic field \(\vect B\) then the incompressible MHD equations are
\begin{eqnarray}
 \rho(\partial_{t} \bu + \bu \cdot \nabla \bu) &=& - \nabla p + \mu \Delta \bu + \bB\cdot \nabla \bB \ ,\\ 
 \nabla \cdot \bu & =& 0\\ \nonumber
   \partial_{t} \bB + \bu \cdot \nabla \bB &=& - \nabla q + \eta \Delta \bB + \bB\cdot \nabla \bu \ ,\\ 
    \nabla \cdot \bB & =& 0 \nonumber
\end{eqnarray}
where \(\rho\) is the density \(\mu\) the viscosity, \(\eta\) resistivity, and pressure \(p\).


The total magnetic field can be split into two parts: \( \mathbf{B} = \mathbf{B_0} + \mathbf{b} \) (mean + fluctuations). The above equations become in terms of Els\"asser variables (\(\mathbf{z}^{\pm} =  \mathbf{u} \pm \mathbf{b} \)) 
\begin{eqnarray}
\frac{\partial {\mathbf{z}^{\pm}}}{\partial t}\mp\left(\mathbf {B}_0\cdot{\mathbf \nabla}\right){\mathbf z^{\pm}} + \left({\mathbf z^{\mp}}\cdot{\mathbf \nabla}\right){\mathbf z^{\pm}} = -{\mathbf \nabla}p 
+ \nu_+ \nabla^2 \mathbf{z}^{\pm} + \nu_- \nabla^2 \mathbf{z}^{\mp} 
\end{eqnarray}
where \( \nu_\pm = \nu \pm \eta \).

The important non-dimensional parameters for MHD are \(Re = U L /\nu \) and the magnetic Re \( Re_M = U L /\eta \).


\subsection{Adaptive Lagrangian-Eulerian (ALE)}

We consider unsteady incompressible flow in a domain with moving boundaries:
\begin{eqnarray} \label{eq:mhd1}
\frac{\partial\mathbf u}{\partial t}&=&-\nabla p +\frac{1}{Re}\nabla\cdot(\nabla + \nabla^T)\mathbf u  + NL,\\
 \nabla \cdot \mathbf u &= &0 
\end{eqnarray}
Here, \(NL\) represents the quadratic nonlinearities from the convective term.

Our free-surface hydrodynamic formulation is based upon the arbitrary 
Lagrangian-Eulerian (ALE) formulation described in \cite{ho89}.
Here, the domain \(\Omega(t)\) is also an unknown.  As with the velocity,
the geometry \(\vect x\) is represented by high-order polynomials.
For viscous free-surface flows,
the rapid convergence of the high-order surface approximation to the 
physically smooth solution minimizes surface-tension-induced stresses
arising from non-physical cusps at the element interfaces, 
where only \(C^0\) continuity is enforced.  
The geometric deformation is specified by a mesh velocity \(\vect w := \dot{\vect x}\)
that is essentially arbitrary, provided that \(\vect w\) satisfies the kinematic
condition \(\vect w \cdot \hat{\vect n}|^{}_{\Gamma} = \vect u \cdot \hat{\vect n}|^{}_{\Gamma}\),
where \(\hat{\vect n}\) is the unit normal at the free surface \(\Gamma(x,y,t)\).
The ALE formulation provides a very accurate description of the free
surface and is appropriate in situations where wave-breaking does not occur.

To highlight the key aspects of the ALE formulation, we introduce
the weighted residual formulation of (\ref{eq:mhd1}):
{\em Find \((\bu,p) \in X^N \times Y^N\) such that:}
\begin{eqnarray} \label{eq:wrt1}
\dd{}{t} (\vect v,\vect u) = (\nabla \cdot \vect v,p) - \frac{2}{Re}(\nabla \vect v,\vect S)
+(\vect v,N\!L) + c(\vect v,\vect w,\vect u),
\qquad
(\nabla \cdot \vect u,q) = 0,
\end{eqnarray} 
for all test functions \((\vect v,q) \in X^N \times Y^N\).
Here \((X^N,Y^N)\) are the compatible velocity-pressure approximation 
spaces introduced in \cite{mapa89}, 
\((.,.)\) denotes the inner-product
\((\vect f,\vect g) := \int_{\Omega(t)} \vect f \cdot \vect g \,dV\),
and 
\(\vect S\) is the stress tensor 
\(S_{ij}^{} := \frac{1}{2}( \pp{u_i}{x_j} + \pp{u_j}{x_i} )\).
For simplicity, we have neglected the surface tension term.
A new term in (\ref{eq:wrt1}) is the trilinear form
involving the mesh velocity
\begin{eqnarray} \label{eq:trilin}
c(\vect v,\vect w,\vect u) :=
\int_{\Omega(t)}^{}
\sum_{i=1}^3 
\sum_{j=1}^3 v_i^{} \pp{w_j^{} u_i^{}}{x_j^{}} \,dV,
\end{eqnarray} 
which derives from the Reynolds transport theorem when
the time derivative is moved outside the bilinear form \((\vect v,\vect u_t^{})\).
The advantage of (\ref{eq:wrt1}) is that it greatly simplifies the
time differencing and avoids grid-to-grid interpolation as the domain
evolves in time.  With the time derivative outside of the integral, 
each bilinear or trilinear form involves functions at a specific time,
\(t^{n-q}\), integrated over \(\Omega(t^{n-q})\).
For example, with a second-order backward-difference/extrapolation scheme,
the discrete form of (\ref{eq:wrt1}) is
\begin{eqnarray} \label{eq:bdk}
\frac{1}{2 \dt}\left[ 
 3 (\vect v^n,\vect u^n)^n
-4 (\vect v^{n-1},\vect u^{n-1})^{n-1}
 + (\vect v^{n-2},\vect u^{n-2})^{n-2} \right]
= L^n (\vect u) + 
2 \widetilde{N\!L}^{n-1}
- \widetilde{N\!L}^{n-2}.
\end{eqnarray} 
Here, \(L^n(\vect u)\) accounts for all {\em linear} terms in (\ref{eq:wrt1}),
including the pressure and divergence-free constraint, which are evaluated
implicitly (i.e., at time level \(t^n\), on \(\Omega(t^n)\)), and
\(\widetilde{N\!L}^{n-q}\) accounts for all {\em nonlinear} terms, including
the mesh motion term (\ref{eq:trilin}), at time-level \(t^{n-q}\).
The superscript on the inner-products \((.,.)^{n-q}\) indicates 
integration over \(\Omega(t^{n-q})\). 
The overall time advancement is as follows.  
The mesh position \(\vect x^n \in \Omega(t^n)\) is computed
explicitly using \(\vect w^{n-1}\) and \(\vect w^{n-2}\);
the new mass, stiffness, and gradient operators involving integrals
and derivatives on \(\Omega(t^n)\) are computed;  
the extrapolated right-hand-side terms are evaluated; and 
the implicit linear system is solved for \(\vect u^n\).   
Note that it is only the {\em operators} that are updated,
not the {\em matrices}.  Matrices are never formed in Nek5000 and
because of this, the overhead for the moving domain formulation
is very low.


%\section{Nek5000 code layout}
%\input{code_layout}

\chapter{Quick start}\label{ch:quick_start}
\input{down_build}

As a first example, we consider the eddy problem due to Walsh 
\footnote{O. Walsh, ``Eddy solutions of the Navier-Stokes equations,''
{\em The NSE II-Theory and Numerical Methods}, J.G. Heywood, K. Masuda, 
R. Rautmann, and V.A. Solonikkov, eds., Springer, pp.  306--309 (1992)}.
To get started, execute the following commands,
\begin{verbatim}
cd
mkdir eddy
cd eddy
cp ~/nek5_svn/examples/eddy/* .
cp ~/nek5_svn/trunk/nek/makenek .
\end{verbatim}

{\bf Modify {\tt makenek}.}

If you do not have {\tt mpi} installed on your system, edit {\tt makenek},
uncomment the {\tt IFMPI="false"} flag, and change the Fortran and C
compilers according to what is available on your machine.  (Most any
Fortran compiler save g77 or g95 will work.)

Nek5000 is written in F77 which has implicit typesetting as default. This means in practice that if the user defines a new variable in the user file and forgets to define its type explicitly then variable beginning with a character from I to N, its type is {\tt INTEGER}. Otherwise, it is {\tt REAL}. 

This common type of mistake for a beginner can be avoided using a warning flag {\tt -Wimplicit}. This flag warns whenever a variable, array, or function is implicitly declared. Has an effect similar to using the IMPLICIT NONE statement in every program unit. 

Another useful flag may {\tt -mcmodel} which allows for arrays of size larger than 2GB. This option tells the compiler to use a specific memory model to generate code and store data. It can affect code size and performance. If your program has global and static data with a total size smaller than 2GB, {\tt -mcmodel=small} is sufficient. Global and static data larger than 2GB requires {\tt -mcmodel=medium} or {\tt -mcmodel=large}.


If you have {\tt mpi} installed on your system or have made the prescribed
changes to makenek, the eddy problem can be compiled as follows


{\bf Compiling nek.}
{\tt makenek eddy\_uv} 

\noindent
If all works properly, upon comilation the executable {\tt nek5000} will be generated using {\tt eddy\_uv.usr} to provide
user-supplied initial conditions and analysis.  Note that if you encountered
a problem during a prior attempt to build the code you should type

{\tt makenek clean;}  

{\tt makenek eddy\_uv} 

\noindent
Once compilation is successful, start the simulation by typing
 

{\bf Running a case:}
{\tt nekb eddy\_uv } 

which runs the executable in the background ({\tt nekb}, as opposed to {\tt
nek}, which will run in the foreground).  
If you are running on a multi-processor machine that supports MPI, you
can also run this case via

{\bf A parallel run:}
{\tt nekbmpi eddy\_uv 4}

\noindent
which would run on 4 processors.    If you are running on a system
that supports queuing for batch jobs (e.g., pbs), then the following
would be a typical job submission command

%% \marginlabel{\bf Running with pbs:}
{\tt nekpbs eddy\_uv 4}

In most cases, however, the details of the nekpbs script would need
to be modified to accommodate an individual's user account, the 
desired runtime and perhaps the particular queue.   Note that the
scripts {\tt nek, nekb, nekmpi, nekbmpi,} etc. perform some essential
file manipulations prior to executing {\tt nek5000}, so it is important
to use them rather than invoking {\tt nek5000} directly.


To check the error for this case, type
\begin{verbatim}
grep -i err eddy_uv.log | tail
\end{verbatim}
or equivalently
\begin{verbatim}
grep -i err logfile | tail
\end{verbatim}
where, because of the {\tt nekb} script, {\tt logfile} is 
linked to the {\tt .log} file of the given simulation. 
If the run has completed, the above {\tt grep} command should yield lines like
\scriptsize
\begin{verbatim}
 1000  1.000000E-01  6.759103E-05  2.764445E+00  2.764444E+00  1.000000E+00  X err
 1000  1.000000E-01  7.842019E-05  1.818632E+00  1.818628E+00  3.000000E-01  Y err
\end{verbatim}
\normalsize
which gives for the $x$- and $y$-velocity components the 
step number, the physical time, the maxiumum error, the maximum exact
and computed values and the mean (bulk) values.

A common command to check on the progress of a simulation is
\begin{verbatim}
grep tep logfile | tail
\end{verbatim}
which typically produces lines such as
\scriptsize
\begin{verbatim}
Step    996, t= 9.9600000E-02, DT= 1.0000000E-04, C=  0.015 4.6555E+01 3.7611E-02
\end{verbatim}
\normalsize
indicating, respectively, the step number, the physical time, the
timestep size, the Courant (or CFL) number, the cumulative wall clock time (in seconds)
and the wall-clock time for the most recent step.   Generally, one would 
adjust $\dt$ to have a CFL of $\sim$0.5.  


%See Section \ref{sec:timestepping} for a comprehensive discussion of timestep selection.

\section{Viewing the First 2D Example}

The preferred mode for data visualization and analysis with Nek5000 is
to use VisIt.  For a quick
peek at the data, however, we list a few commands for the native Nek5000 
postprocessor.   Assuming that the {\tt maketools} script has been executed
and that {\tt /bin} is in the execution path, then typing 

\noindent
{\tt postx} 

\noindent
in the working directory should open a new window with a sidebar menu.
With the cursor focus in this window (move the cursor to the window and
left click), hit {\tt return} on the keyboard accept the default session name and click {\sc plot} with the left mouse button.  This should bring up
a color plot of the pressure distribution for the first output file
from the simulation (here, {\tt eddy\_uv.fld01}), which contains the
geometry, velocity, and pressure.  

Alternatively one can use the script \textit{visnek}, to be found in {\tt /scripts}. It is sufficent to run 

\noindent
{\tt visnek eddy\_uv}\textit{ (or the name of your session)}

to obatain a file named {\tt eddy\_uv.nek5000} which can be recognized in VisIt \footnote{https://wci.llnl.gov/simulation/computer-codes/visit/}


\begin{comment}
To see the vorticity at the final time, load the last output file,
{\tt eddy\_uv.fld12}, by clicking/typing the following in the postx window:
\begin{tabular}{r l l l}
  & {\bf click} \hspace*{1in} &{\bf type} \hspace*{1in} & {\bf comment} \\ \hline
1.& SET TIME         & 12 & load fld12 \\
2.& SET QUANTITY \\
3.& VORTICITY \\
4.& PLOT 
\end{tabular}
\end{comment}

{\bf Plotting the error:}
For this case, the error has been written to {\tt
eddy\_uv.fld11} by making a call to {\tt outpost()} in the {\tt userchk()}
routine in {\tt eddy\_uv.usr}.  The error in the velocity components
is stored in the velocity-field locations and can be viewed with 
postx, or VisIt as before.

\begin{comment}
through the following sequence: 
\begin{tabular}{r l l l}
  & {\bf click} \hspace*{1in} &{\bf type} \hspace*{1in} & {\bf comment} \\ \hline
1.& SET TIME         & 11 & load fld11 \\
2.& SET QUANTITY \\
3.& VELOCITY \\
4.& MAGNITUDE \\
5.& PLOT  \\
\end{tabular}
\end{comment}

\subsection{Modifying the First Example}

A common step in the Nek5000 workflow is to rerun with a higher
polynomial order.   Typically, one runs a relatively low-order case
(e.g., {\tt lx1}=5) for one or two flow-through times and then uses
the result as an initial condition for a higher-order run
(e.g., {\tt lx1}=8).  We illustrate the procedure with the 
{\tt eddy\_uv} example.

Assuming that the contents of {\tt nek5\_svn/trunk/tools/scripts}
are in the execution path, begin by typing
\begin{verbatim}
cp eddy_uv eddy_new
\end{verbatim}
which will copy the requisite {\tt eddy\_uv} case files
to {\tt eddy\_new}.  
Next, edit {\tt SIZE} and change the two lines defining
{\tt lx1} and {\tt lxd} from
\begin{verbatim}
      parameter (lx1=8,ly1=lx1,lz1=1,lelt=300,lelv=lelt)
      parameter (lxd=12,lyd=lxd,lzd=1)
\end{verbatim}
to
\begin{verbatim}
      parameter (lx1=12,ly1=lx1,lz1=1,lelt=300,lelv=lelt)
      parameter (lxd=18,lyd=lxd,lzd=1)
\end{verbatim}
Then recompile the source by typing
\begin{verbatim}
makenek eddy_new
\end{verbatim}

Next, edit {\tt eddy\_new.rea} and change the line 
\begin{verbatim}
            0 PRESOLVE/RESTART OPTIONS  *****
\end{verbatim}
(found roughly 33 lines from the bottom of the file) to
\begin{verbatim}
            1 PRESOLVE/RESTART OPTIONS  *****
eddy_uv.fld12
\end{verbatim}
which tells nek5000 to use the contents of {\tt eddy\_uv.fld12}
as the initial condition for {\tt eddy\_new}.
The simulation is started in the usual way:
\begin{verbatim}
nekb eddy_new
\end{verbatim}
after which the command
\begin{verbatim}
grep err logfile | tail
\end{verbatim}
will show a much smaller error ($\sim 10^{-9}$) than the {\tt lx1=8}
case. 

Note that one normally would not use a restart file for the {\em eddy}
problem, which is really designed as a convergence study.  The purpose here, however, was two-fold, namely,
to illustrate a change of order and its impact on the error, and to
demonstrate the frequently-used restart procedure. However for a higher order timestepping scheme an accurate restart would require a number of field files of the same size (+1) as the order of the multistep scheme


\chapter{User files}
\section{Case set-up .usr}

%Nek5000 consists of three principal modules:  the preprocessor
%{\tt prex}, the solver {\tt nek5000}, and the postprocessor {\tt postx}.
%{\tt prex} and {\tt postx} are based upon an X-windows GUI.  

Each simulation is defined by three files, the .rea file, the .usr file,
and the SIZE file.  In addition, there is a derived .map file that is
generated from the .rea file by running {\tt genmap} which will determine how the elements will be split across processors in the case of a parallel run.
%Suppose you are doing a calculation called ``shear.''
%The key files defining the problem would be shear.rea, shear.map, and shear.usr.
SIZE controls (at compile time) the polynomial degree used in the simulation,
as well as the space dimension \(d=2\) or 3.

The SESSION.NAME file contains the current run, it must provide the name of the .rea file and the path to it.  It does not however need to correspond to an .usr file of an identical name. This allows for different test cases (.usr files) that use the same geometry and boundary conditions (.rea files).

This chapter provides an introduction to the basic files required
to set up a Nek5000 simulation.

%\marginlabel{\bf {\tt .usr} files:}
\subsection{Contents of .usr file}

The most important interface to Nek5000 is the
set of Fortran subroutines that are contained in the {\tt .usr} file.
This file allows direct access to all runtime variables.
Here, the user may specify spatially varying properties
(e.g., viscosity), volumetric heating sources, body forces, and so forth.
One can also specify arbitrary initial and boundary conditions through
the routines {\tt useric()} and {\tt userbc()}.
The routine {\tt userchk()} allows the user to interrogate the solution
at the end of each timestep for diagnostic purposes.   The {\tt .usr}
files provided in the {\tt /examples/... } directories illustrate
several of the more common analysis tools.  For instance, there are utilities
for computing the time average of \(u\), \(u^2\), etc. so that one can
analyze mean and rms distributions with the postprocessor.  There are
routines for computing the vorticity or the scalar \(\lambda_2\) 
for vortex identification, and so forth.


\subsubsection*{Routines in .usr file}

The routine {\tt uservp} specifies the variable properties of the governing equations.
This routine is called once per processor, and once per discrete point therein. 
%For each field (fluid {\tt ifield.eq.1}, passive scalar {ifield.eq.2}(3,4..)) we have to specify two %components, {\tt utrans} corresponding to the transport term (\(\rho\) in Eq.\ref{eq:ns_momentum}, \(\rhocp\) in %Eq
\\

\begin{tabular}{ l|l|l|l }
   \hline
   Equation & {\tt udiff} & {\tt utrans} & {\tt ifield} \\ \hline \hline
   Momentum Eq.\ref{eq:ns_momentum} & \(\rho\) & \(\mu\) & 1 \\ 
   Energy Eq.\ref{eq:energy} & \(\rho c_p\) & \(k\) & 2\\ 
   Passive scalar Eq.\ref{eq:pass_scal} &\((\rho c_p)_i\) & \(k_i\)& i-1\\
   \hline
\end{tabular}


\begin{lstlisting}
      subroutine uservp (ix,iy,iz,eg)
      include 'SIZE'
      include 'TOTAL'
      include 'NEKUSE'
      
      integer iel
      iel = gllel(eg)

      udiff =0.
      utrans=0.
      
      return
      end
\end{lstlisting}

The routine {\tt userdat} is called right after the geometry is loaded into NEK5000 and prior to the distribution of the GLL points. This routine is called once per processor but for all the data on that processor. At this stage the elements can be modified as long as the topology is preserved. It is also possible to alter the type of boundary condition that is initially attributed in the {\tt .rea} file, as illustrated below (the array {\tt cbc(face,iel,field}) contains the boundary conditions per face and field of each element). Note the spacing allocated to each BC string is of three units.

\begin{lstlisting}
      subroutine usrdat
      include 'SIZE'
      include 'TOTAL'
      include 'NEKUSE'
      integer iel,f

      do iel=1,nelt  !  Force flux BCs
      do f=1,2*ndim
         if (cbc(f,iel,1).eq.'W  ') cbc(f,iel,2) = 'f  ' ! flux BC for temperature
      enddo
      enddo
   
      return
      end
\end{lstlisting}

The routine {\tt usrdat2} is called after the GLL points were distributed and allows at this point only for affine transformations of the geometry.
\begin{lstlisting}
      subroutine usrdat2
      include 'SIZE'
      include 'TOTAL'

      return
      end
\end{lstlisting}

The routine {\tt userf} is called once for each point and provides the force term in Eq.\ref{eq:ns_momentum}. Not that according to the dimensionalization in Eq.\ref{eq:ns_momentum} the force term \(\vect f\) is in fact multiplied by the density \(\rho\).
\begin{lstlisting}
      subroutine userf  (ix,iy,iz,eg)
      include 'SIZE'
      include 'TOTAL'
      include 'NEKUSE'

      ffx = 0.0
      ffy = 0.0
      ffz = 0.0

      return
      end
\end{lstlisting}

Similarly to {\tt userf} the routine {\tt userq} provides the force term in Eq.\ref{eq:energy} and the subsequent passive scalar equations according to Eq.\ref{eq:pass_scal}.
\begin{lstlisting}	
      subroutine userq  (ix,iy,iz,eg)
      include 'SIZE'
      include 'TOTAL'
      include 'NEKUSE'
      
      qvol   = 0.

      return
      end
      \end{lstlisting}
      
      The boundary conditions are assigned in {\tt userbc} for both the fluid, temperature and all other scalars. An extensive list of such possible boundary conditions is available in Section.~\ref{sec:boundary}. 
      \begin{lstlisting}
      subroutine userbc (ix,iy,iz,iside,ieg)
      include 'SIZE'
      include 'TOTAL'
      include 'NEKUSE'

      ux=0.0
      uy=0.0
      uz=0.0
      temp=0.0
      flux = 1.0
      
      return
      end
\end{lstlisting}

Initial conditions are attributed in {\tt useric} similarly to the boundary conditions
\begin{lstlisting}
      subroutine useric (ix,iy,iz,ieg)
      include 'SIZE'
      include 'TOTAL'
      include 'NEKUSE'
   
      uy=0.0
      ux=0.0
      uz=1.0

      return
      end
      
\end{lstlisting}
The routine {\tt userchk} is called once per processor after each timestep (and once after the initialization is finished). This is the section where the solution can be interrogated and subsequent changes can be made.
\begin{lstlisting}
      subroutine userchk
      include 'SIZE'
      include 'TOTAL'
      include 'NEKUSE'

      call outpost(vx,vy,vz,pr,t,'ext')
           
      return
      end
      \end{lstlisting}
      
The routine {\tt usrdat3} is not widely used, however it shares the same properties with {\tt usrdat2}.
\begin{lstlisting}
      subroutine usrdat3
      include 'SIZE'
      include 'TOTAL'
c
      return
      end
\end{lstlisting}

Nek5000 can solve the dimensional or non-dimensional equations by setting the following parameters

\begin{table}

\begin{tabular}{ l|l| }
   \hline
   Dimensional parameters & Non-dimensional parameters\\ \hline \hline
{\tt p1}=\(\rho\)      &      {\tt p1}=1\\
{\tt p2}=\(\nu\)       &      {\tt p2}=1/Re (-Re)\\
{\tt p7}=\(\rho C_p\)  &      {\tt p7}=1\\
{\tt p8}=\(k\)         &      {\tt p8}=1/Pe (-Pe)\\
   \hline
\end{tabular}
\end{table}

alternatively the variable properties can be set in the USERVP routine.

 
\textbf{What is a SESSION file?}

To run NEK5000, each simulation must have a SESSION.NAME file. This file is read in by the code and gives the path to the relevant files describing the structure and parameters of the simulation. The SESSION.NAME file is a file that contains the name of the simulation and the full path to supporting files. For example, to run the eddy example from the repository, the SESSION.NAME file would look like:

\begin{verbatim}
eddy_uv\\
/homes/user\_ name/nek5\_ svn/examples/eddy/ 
\end{verbatim}


\section{Problem size file SIZE}
\input{size_intro}
\section{Geometry and Parameters file .rea}


%\marginlabel{\bf {\tt .rea} files:}
The {\tt .rea} file consists of several sections:
\subsubsection*{ Parameters and logical switches}
\begin{description}
\item{\bf parameters} These control the runtime parameters such as viscosity,
    conductivity, number of steps, timestep size, order of the timestepping,
    frequency of output, iteration tolerances, flow rate, filter strength,
    etc.   There are also a number of free parameters that the user can
    use as handles to be passed into the user defined routines in the .usr file.
\item{\bf passive scalar data} This information can be specified also in the \texttt{.uservp} routine in the .usr file. If specified in the .rea file then the coefficients for the conductivity term are listed in ascending order for passive scalars ranging \texttt{1..9} followed by the values for the \(\rho c_p\) coefficients.
\begin{verbatim}
4  Lines of passive scalar data follows 2 CONDUCT; 2 RHOCP
   1.00000       1.00000       1.00000       1.00000       1.00000
   1.00000       1.00000       1.00000       1.00000
   1.00000       1.00000       1.00000       1.00000       1.00000
   1.00000       1.00000       1.00000       1.00000
\end{verbatim}
\item{\bf logicals} These determine whether one is computing a steady or unsteady
  solution, whether advection is turned on, etc.
\end{description}

Next we have the logical switches as follow. a detailed explanation to be found in Sec:\ref{sec:switches}
\begin{verbatim}
           13  LOGICAL SWITCHES FOLLOW
  T     IFFLOW
  T     IFHEAT
  T     IFTRAN
  T T F F F F F F F F F IFNAV & IFADVC (convection in P.S. fields)
  F F T T T T T T T T T T IFTMSH (IF mesh for this field is T mesh)
  F     IFAXIS
  F     IFSTRS
  F     IFSPLIT
  F     IFMGRID
  F     IFMODEL
  F     IFKEPS
  F     IFMVBD
  F     IFCHAR
\end{verbatim}
\subsubsection*{Mesh and boundary condition info} 
\begin{description}
\item{\bf geometry} The geometry is specified in an arcane format specifying
    the \(xyz\) locations of each of the eight points for each element,
    or the \(xy\) locations of each of the four points for each element in 2D.
A line of the following type may be encountered at the beginning of the mesh section of the area file.    
\begin{verbatim}
3.33333       3.33333     -0.833333      -1.16667     XFAC,YFAC,XZERO,YZERO
\end{verbatim}
This part is to be read by PRENEK and provides the origin of the system of coordinates \texttt{XZERO;YZERO} as well as the size of the cartesian units \texttt{XFAC;YFAC}. This one line has no impact on the mesh as being read in NEK. 

The header of the mesh data may have the following representation
\begin{center}
\begin{verbatim} **MESH DATA** 6 lines are X,Y,Z;X,Y,Z. Columns corners 1-4;5-8
      226  3         192           NEL,NDIM,NELV
     ELEMENT           1 [    1A]    GROUP    0
     \end{verbatim}
     \end{center}
The header states first how many elements are available in total (\(226\)), what dimension is the the problem (here three dimensional), and how many elements are in the flow mesh (\(192\)). 

%\begin{center}
%     \begin{tabular}{l|l|l|l}
%  0.000000E+00 & 0.171820E+00 & 0.146403E+00 & 0.000000E+00 \\
%  0.190000E+00 & 0.168202E+00 & 0.343640E+00 & 0.380000E+00 \\
%  0.000000E+00 & 0.000000E+00 & 0.000000E+00 & 0.000000E+00 \\
%  0.000000E+00 & 0.171820E+00 & 0.146403E+00 & 0.000000E+00 \\
%  0.190000E+00 & 0.168202E+00 & 0.343640E+00 & 0.380000E+00  \\
%  0.250000E+00 & 0.250000E+00 & 0.250000E+00 & 0.250000E+00  \\
%  \end{tabular}
%\end{center}

\begin{table}
\subfloat[Descriptor]{
\begin{tabular}{l c c c c}
%\fontsize 
  \(\texttt{Face \{1,2,3,4\}}\)&&&&\\
  \(x_{1,\ldots,4}=\)& 0.000000E+00 & 0.171820E+00 & 0.146403E+00 & 0.000000E+00 \\
  \(y_{1,\ldots,4}=\)&0.190000E+00 & 0.168202E+00 & 0.343640E+00 & 0.380000E+00 \\
  \(z_{1,\ldots,4}=\)&0.000000E+00 & 0.000000E+00 & 0.000000E+00 & 0.000000E+00 \\
  \(\texttt{Face \{5,6,7,8\}}\)&&&&\\
  \(x_{5,\ldots,8}=\)&0.000000E+00 & 0.171820E+00 & 0.146403E+00 & 0.000000E+00 \\
  \(y_{5,\ldots,8}=\)&0.190000E+00 & 0.168202E+00 & 0.343640E+00 & 0.380000E+00  \\
  \(z_{5,\ldots,8}=\)&0.250000E+00 & 0.250000E+00 & 0.250000E+00 & 0.250000E+00  
  \end{tabular}} 
%\subfloat[Element]{\raisebox{-10pt}{\vspace{0.5cm}\includegraphics[scale=0.4]{Figs/3dcube}\label{fig:3dcube}}}
%{\begin{minipage}[c][1\width]{0.4\textwidth}\centering \includegraphics[width=0.4\textwidth]{Figs/3dcube} \end{minipage}}
\caption{Geometry description in .rea file}
\end{table}
\normalsize

\begin{figure}
\includegraphics[scale=0.5]{Figs/3dcube}
\caption{Geometry description in .rea file (sketch of one element ordering)}
\end{figure}

\item{\bf curvature} 
     This section describes the deformation for elements that are curved.
     Currently-supported curved side or edge definitions include ``C''
     for circles, ``s'' for spheres, and ``m'' for midside-node positions
     associated with quadratic edge displacement. If no curved data is available the section remains empty.
     Example
     
     The section header may look like this 
     \begin{center}
     \texttt{640 Curved sides follow IEDGE,IEL,CURVE(I),I=1,5, CCURVE} 
     \end{center}
     and the data is stored as follows
  \footnotesize   
     \begin{center}
\begin{tabular}{ l|l|l|l|l }
   \hline
 \texttt{IEDGE}& \texttt{IEL} &\texttt{CURVE(12,1,IEL)} &\texttt{CURVE(12,2..5,IEL)}&\texttt{CCURVE(12,IEL)} \\ \hline \hline
  3&  1 &  1.0000  &      0.0000 &    C \\
  7 & 1 &  1.0000  &      0.0000 &    C\\
   \hline
\end{tabular}   
\end{center}
\normalsize
     The array \texttt{CCURVE} (char curve) holds a character denoting the type of curved boundary, while the array \texttt{CURVE} holds the actual information about the curved boundary. There are up to five available components in the \texttt{CURVE} array in case more information is needed by other implementations, that do not represent the default. We may have
     \begin{itemize}
     \item 'C' stands for circle and is given by the radius of the circle, thus filling in only the first component of the \texttt{CURVE(12,1,NEL)} 
     \item 'S' stands for sphere and is given by the radius and the center of the sphere, thus filling the first 4 components of the \texttt{CURVE(12,1\ldots 4,NEL)}
     \item 'M' is given by the coordinates of the midside-node, thus filling the first 3 components of the \texttt{CURVE(12,1\ldots 3,NEL)}, and leads to a second order reconstruction of the face.
     \end{itemize}
Both 'C' and 'S' types allow for a surface of as high order as the polynomial used in the spectral method, since they have an underlying analytical description, any circle arc can be fully determined by the radius and end points. However for the 'M' curved element descriptor the surface can be reconstructed only up to second order. This can be later updated to match the high-order polynomial after the GLL points have been distributed across the boundaries. In .usrdat2 the user can move the geometry to match the intended surface, followed by a call to the subroutine 'fixgeom' which can realign the point distribution in the interior of the element.

\item{\bf boundary conditions} 
     Boundary conditions (BCs) are specified for each face of each element,
     for each {\tt field} (velocity, temperature, passive scalar \#1, etc.).
     A common BC is {\tt P}, which indicates that an element face is 
     connected to another element to establish a periodic BC.   Many of the 
     BCs support either a constant specification or a user defined
     specification which
     may be an arbitrary function.   For example, a constant Dirichlet
     BC for velocity is specified by {\tt V}, while a user defined BC
     is specified by {\tt v}.   This upper/lower-case distinction is 
     used for all cases.   There are about 70 different types of boundary
     conditions in all, including free-surface, moving boundary, heat flux,
     convective cooling, etc.
     
      The section header may look like this 
     \begin{center}
     \texttt{ ***** FLUID   BOUNDARY CONDITIONS *****}

     \end{center}
     and the data is stored as follows
      \footnotesize   
     \begin{center}
\begin{tabular}{ l|l|l|l|l|l }
   \hline
 \texttt{CBC}& \texttt{IEL} &\texttt{IEDGE} &\texttt{CONN-IEL}&\texttt{CONN-IEDGE} & redundant\\ \hline \hline
  E   & 1 & 1 &  4.00000   &    3.00000  &     0.00000      \\
   ..   & .. & .. &  ..   &   .. &    ..      \\
   W  &  5 & 3 &  0.00000  &     0.00000  &     0.00000    \\
    ..   & .. & .. & ..   &   ..  &    ..      \\
   P  &  5 & 5  & 149.000 &      6.00000  &     0.00000 \\
   \hline
\end{tabular}   
\end{center}
\normalsize
     
\end{description}

\subsubsection*{ Output info} 
\begin{description}
\item{\bf restart conditions} 

     Here, one can specify a file to use as an initial condition.
     The initial condition need not be of the same polynomial order
     as the current simulation.   One can also specify that, for example,
     the velocity is to come from one file and the temperature from another.
     The initial time is taken from the last specified restart file, but 
     this can be overridden.
\item{\bf History points}

The following section defines history points in the {\tt .rea} file, see example {\tt vortex/r1854a.rea}, or {\tt shear4/shear4.rea}
\begin{verbatim}
0 PACKETS OF DATA FOLLOW\\
***** HISTORY AND INTEGRAL DATA *****\\
    56 POINTS. H code, I,J,H,IEL \\
UVWP    H     31     31   1   6\\
UVWP    H     31     31   31  6\\
UVWP    H     31     31   31  54\\
 "      "      "      "    "   "\\
\end{verbatim}

The {\tt "56 POINTS"} line needs to be followed by 56 lines of the type shown. However, in each of the following lines, which have the {\tt UVWP} etc., location is CRUCIAL, it
must be layed out exactly as indicated above\footnote{these lines contain character strings, they use formatted reads}, it is therefore advisable to refer to the examples {\tt vortex, shear4}.  If you want to pick points close to the center of element 1 and are running with lx1=10, say, you might choose {\tt UVWP H 5 5 5 1}. \footnote{the indicated point would really be at the middle of the element only if lx1=9}

The UVWP tells the code to write the 3 velocity components and pressure to the .sch file at
each timestep (or, more precisely, whenever {\tt mod(istep,iohis)=0}, where {\tt iohis=param(52))}.
Note that if you have more than one history point then they are written sequentially at each
timestep. Thus 10 steps in the first example with {\tt param(52)=2} would write {\tt (10/2)*56 = 280}
lines to the .sch file, with 4 entries per line. The "H" indicates that the entry corresponds to a requested history point. A note of caution: if the {\tt ijk} values (5 5 5 in the preceding example line) exceed {\tt lx1,ly1,lz1} of your SIZE file, then they are truncated to that value. For example, if {\tt lx1=10} for the data at the top (31 31 31) then the code will use {\tt ijk} of (10 10 10), plus the given element number, in identifying the history point. It is often useful to set {\tt ijk} to large values (i.e., > {\tt lx1}) because the endpoints of the spectral element mesh are invariant when {\tt lx1} is changed. 

\begin{comment}
7. A difficulty with the current nek history point specification is finding the requisite ijke (e=element
number) values that correlate to the point of interest. There is a way to do this in postx that
is relatively painless, but this is not useful for very large problems. (The approach is:
SET PLOT FORMAT
SCALAR
VALUES
PLOT
Follow the instructions and for each point requested, postx will write to the screen lines that
are similar to the above, ready to be pasted into the .rea file.)
8. When you run nek, it will write the coordinate information to the logfile on the first timestep
so that you can verify the point locations.
\end{comment}
\item{\bf output specifications} 
     Outputs are discussed in a separate section below.
\end{description}


\noindent
It is important to note that Nek5000 currently supports two input file
formats, ascii and binary.   The {\tt .rea} file format
described above is ascii.  For the binary format, all sections
of the .rea file having storage requirements that scale with 
number of elements (i.e., geometry, curvature, and boundary 
conditions) are moved to a second, {\tt .re2}, file and
written in binary.   The remaining sections continue to 
reside in the {\tt .rea} file.   The distinction between
the ascii and binary formats is indicated in the {\tt .rea}
file by having a negative number of elements.
There are converters, {\tt reatore2} and {\tt re2torea}, in the Nek5000
tools directory to change between formats.   The binary file
format is the default and important for {\tt I/O} performance when the
number of elements is large ( \(>\) 100000, say).

    
\subsection{Parameters}
\begin{itemize}  
\item $\rho$, the density, is taken to be time-independent and
  constant; however, in a multi-fluid system
  different fluids can have different value of constant density.
\item $\mu$, the dynamic viscosity can vary arbitrarily in
  time and space; it can also be a function of temperature
  (if the energy equation is included) and strain rate
  invariants (if the stress formulation is selected).
\item $\sigma$, the surface-tension coefficient can vary
  arbitrarily in
  time and space; it can also be a function of temperature
  and passive scalars.
\item $\overline{\beta}$, the effective thermal expansion
  coefficient, is
  assumed time-independent and constant.
\item ${\bf f}(t)$, the body force per unit mass term can
  vary with time, space, temperature and passive scalars.
\item $\rho c_{p}$, the volumetric specific heat, can vary
  arbitrarily with time, space and temperature.
\item $\rho L$, the volumetric latent heat of fusion at a front,
  is taken to be time-independent and constant; however,
  different constants can be assigned to different fronts.
\item $k$, the thermal conductivity, can vary with time,
  space and temperature.
\item $q_{vol}$, the volumetric heat generation, can vary with
  time, space and temperature.
\item $h_{c}$, the convection heat transfer coefficient, can vary
  with time, space and temperature.
\item $h_{rad}$, the Stefan-Boltzmann constant/view-factor product,
  can vary with time, space and temperature.
\item $T_{\infty}$, the environmental temperature, can vary
  with time and space.
\item $T_{melt}$, the melting temperature at a front, is taken
  with time and space; however, different melting temperature
  can be assigned to different fronts.
\end{itemize}
  
In the solution of the governing equations together with
the boundary and initial conditions, Nek5000 treats the
above parameters as pure numerical values; their
physical significance depends on the user's choice of units.
The system of units used is arbitrary (MKS, English, CGS,
etc.). However, the system chosen must be used consistently
throughout. For instance, if the equations and geometry
have been non-dimensionalized, the $\mu / \rho$ in the fluid
momentum equation is in fact
the inverse Reynolds number, whereas if the equations are
dimensional, $\mu / \rho$ represents the kinematic viscosity with
dimensions of $length^{2}/time$.
%\begin{comment}
\section{Data Layout}
\label{sec:data_layout}

Nek5000 was designed with two principal performance criteria in mind,
namely, {\em single-node} performance and {\em parallel} performance.

A key precept in obtaining good single node performance was to use,
wherever possible, unit-stride memory addressing, which is realized by
using contiguously declared arrays and then accessing the data in
the correct order.   Data locality is thus central to good serial 
performance.   To ensure that this performance is not compromised
in parallel, the parallel message-passing data model is used, in which
each processor has its own local (private) address space.  Parallel
data, therefore, is laid out just as in the serial case, save that there
are multiple copies of the arrays---one per processor, each containing 
different data.  Unlike the shared memory model, this distributed memory
model makes data locality transparent and thus simplifies the task of
analyzing and optimizing parallel performance.

Some fundamentals of Nek5000's internal data layout are given below.

\begin{enumerate}
\item
Data is laid out as  \(u_{ijk}^e = u(i,j,k,e)\) \\

{\tt   i=1,...,nx1   (nx1 = lx1)} \\
{\tt   j=1,...,ny1   (ny1 = lx1)} \\
{\tt   k=1,...,nz1   (nz1 = lx1} or 1, according to ndim=3 or 2) \\

{\tt   e=1,...,nelv}, where {\tt nelv} \(\le\) {\tt lelv}, and {\tt lelv} is the upper
                 bound on number of elements, {\em per processor}.


\item
 Fortran data is stored in column major order (opposite of C).

\item
 All data arrays are thus contiguous, even when {\tt nelv} \(<\) {\tt lelv}

\item Data accesses are thus primarily unit-stride (see chap.8 of DFM
   for importance of this point), and in particular, all data on
   a given processor can be accessed as, e.g.,


\begin{verbatim}
   do i=1,nx1*ny1*nz1*nelv
      u(i,1,1,1) = vx(i,1,1,1)
   enddo
\end{verbatim}

   which is equivalent but superior to:

\begin{verbatim}
   do e=1,nelv
   do k=1,nz1
   do j=1,ny1
   do i=1,nx1
      u(i,j,k,e) = vx(i,j,k,e)
   enddo
   enddo
   enddo
   enddo
\end{verbatim}


   which is equivalent but vastly superior to:

\begin{verbatim}
   do i=1,nx1
   do j=1,ny1
   do k=1,nz1
   do e=1,nelv
      u(i,j,k,e) = vx(i,j,k,e)
   enddo
   enddo
   enddo
   enddo
\end{verbatim}


\item All data arrays are stored according to the SPMD programming
   model, in which address spaces that are local to each processor
   are private --- not accessible to other processors except through
   interprocessor data-transfer (i.e., message passing).  Thus

\begin{verbatim}
   do i=1,nx1*ny1*nz1*nelv
      u(i,1,1,1) = vx(i,1,1,1)
   enddo
\end{verbatim}

   means different things on different processors and {\tt nelv} may
   differ from one processor to the next.  

\item For the most part, low-level loops such as above are expressed in
   higher level routines only through subroutine calls, e.g.,:

\begin{verbatim}
   call copy(u,vx,n)
\end{verbatim}

   where {\tt n:=nx1*ny1*nz1*nelv}.   Notable exceptions are in places where
   performance is critical, e.g., in the middle of certain iterative
   solvers.

\end{enumerate}


%\subsection{Additional files}

\chapter{Geometry}\label{sec:geom}

\section{Generating a Mesh with Genbox}\label{sec:genbox}
%\subsection{Rectangular geometries}
Genbox is a tool for generating a number of tensor-product boxes, i.e., a set
of \(nelx \times nely \times nelz\) elements, where the locations of the vertices of the
elements are given as one-dimensional arrays of length \(nelx+1\), \(nely+1\), and
\(nelz+1\), respectively.  After running genbox, an output mesh file (.rea) is
created according to specifications in the input .box file.

\subsubsection{Running genbox}

Before running genbox, ensure that the Nek5000 tools in \texttt{Nek5000/tools}
are compiled and up-to-date.  Also ensure that the Nek5000 tools have been
added to your executable path (i.e., the \texttt{\$PATH} variable).  See
section~\ref{ch:quick_start}) for information on compiling the Nek5000 tools.

To run genbox, type \texttt{genbox} on the command line.  You will be prompted
for the filename of the input .box file with the following message:

\begin{verbatim}
Enter the full name of the .box file, including the .box extension
\end{verbatim}

See Section~\ref{sec:box_file} for a description of the .box file format.  A
successful genbox run will create a box.rea file (and a box.re2 file if
requested).  From here, you may use genmap to create a box.map file (see
Section~\ref{sec:genmap} for instructions on using genmap).  At any point in
this process, the box.xxx files may be moved and renamed using the script mvn
in \texttt{Nek5000/bin/mvn}.  

\subsection{The .box file format}\label{sec:box_file}

Genmap uses .box files to specify geometries.  The box file includes a header
that is formatted the same for all geometries; a description that is formatted
for specific geometries; and a footer that is formatted the same for all
geometries.  In the .box file, all lines beginning with the hash symbol
(\texttt{\#}) are comments and will be ignored.  \textbf{Genbox can only read
in 132 characters per line, so any line that exceeds this limit MUST be split
into 2 lines.}

\subsubsection{Header for all geometries}

\begin{description}

  \item[Line 1] supplies the name of an existing .rea file that has the
    appropriate run parameters.  While the parameters can be modified later, it
    is important that the dimension of the .rea file matches the dimension
    desired for the new files.

  \item[Line 2] indicates the spatial dimension. If less than zero, genbox will
    create an ASCII .rea containing the runtime parameters and a binary .re2
    file containing the mesh and boundary data.

  \item[Line 3] specifies the number of fields for this simulation, $nfld$
    (corresponding to velocity, heat transfer, etc.). Note that $nfld$ must
    never be set to zero!

    \begin{itemize}

      \item If $nfld<0$: genbox will create files that solve for only heat +
        $nfld-1$ passive scalars and \textbf{not} for fluid. 

      \item If $nfld>0$: genbox will solve for fluid, heat, and then $nfld-2$
        passive scalars.   
    
      \item If you intend to run MHD, $nfld$ should be a decimal number of the
        format \texttt{X.1} where the integer part \texttt{X} is the number of
        fields and the decimal part \texttt{.1} instructs genbox to look for
        the magnetic field properties.

    \end{itemize}

  \item[Line 4] indicates that a new box is starting and that the following
    lines will describe that box.  All .box files have boundary conditions
    defined at the end, so after defining the geometry-specific fields found in
    the appropriate section, go to the final, "for all geometries" section to
    see how to finish the .box file correctly.

    \begin{itemize}

      \item For normal, box geometries, this can be any string that doesn't
        start with a "c", "C", "m", "M", "y", or "Y"

      \item For annulus geometries with circular-only edges (and the option
        less than 360 degree), this line must start with a "c" or a "C"

      \item For annulus geometries with varying boundary shapes, this line must
        start with a "y" or a "Y"

      \item For geometries with varying segments of x,y,(z) distributions, this
        line must start with a "m" or "M"

    \end{itemize}

\end{description}

\subsubsection{Description for box geometries}

\begin{description}

  \item[Line 5] specifies the number of elements in the $x$ and $y$ (and for
    3D, $z$) directions.  
    
    \begin{itemize}

      \item If these values are negative, genbox will automatically generate
        the element distribution along each axis. 

      \item If these values are positive, the user must explicitly provide the
        element distributions.  

    \end{itemize}
    

  \item[Lines 6, 7, (8 in 3D)] The lines following $nelx$, $nely$, $nelz$
    specify the distribution of elements in each spatial direction Since there
    is a limit of 132 characters per line, it may be necessary to wrap these
    points across various lines.

    \begin{itemize}

      \item If the number of elements was less than zero, the user only
        supplies the start point $x0$, the ending point $xnelx$ and the ratio,
        or relative size, of each element progressing from left to right.  

      \item If the number of elements was greater than zero, each $nelx$ point
        in that direction, i.e., $x0, x1, \ldots, xnelx$ must be specified.

    \end{itemize}

\end{description}

\subsubsection{Description for circular-only annulus geometries}

\begin{description} 
  
  \item[Line 5] Specifies the $x$, $y$, (and for 3D, $z$) center of the
    geometry. Note that this is an annulus geometry, so the $x$ starting point
    cannot equal the center.

  \item[Line 6] The number of elements in each direction $nelx$, $nely$, $nelz$

    \begin{itemize}

      \item If any are negative, genbox will automatically distribute the
        elements for the respective dimensions(s).

      \item If any are positive, genbox will look for user-specified elements
        for the respective dimension(s).

    \end{itemize}

  \item [Lines 7, 8, (9 in 3D)] The element distribution of each dimension.  The
    $y$ dimension is given in degrees. For example, '0 360 1' would be a full
    circle with uniform distribution, and '90 180 1' would be a quarter-circle
    starting at the 90 degree point.

\end{description}

\subsubsection{Description for annulus geometries}

Genbox is currently setup for ascii output only for this option.

\begin{description}

  \item[Line 5] Specifies the number of elements in each direction, $nelx$,
    $nely$, (and for 3D, $nelz$).

    \begin{itemize}

      \item If any are negative, genbox will automatically distribute the
        elements for this dimension.

      \item If any are positive, genbox will look for user-specified elements
        for this dimension.

    \end{itemize}

  \item[Line 6] Coodinates of the center of the annulus, $(x0, y0)$. Since this
    is annulus geometry, x0,y0,z0 must not be the center.

  \item[Line 7] String that specifies the cylinder type for each x-level of the
    annulus.  There must be $nelx+1$ characters in this string.  The available
    characters are:

    \begin{description}
      \item[``c''] curve boundary geometry
      \item[``o''] octagon
      \item[``b''] box (Cartesian)
    \end{description}

    When mixing curve (c) sides and box (b) sides, the elements on each level
    must not overflow into the next.  This will produce an error in genbox.  To
    fix this, the $nelx$ points (or ratio) must be adjusted.  Furthermore,
    box-like sides work best with an even number of $nely$ elements.

  \item[Lines 8, 9, (10 in 3D)] The coordinates of the elements.

    \begin{itemize}

      \item If the user gave a negative dimension, they only need to provide
        $r0$, $r1$, $ratio$ (start, end, ratio) for this dimension

      \item If the user gave a positive dimension, they will need to provide
        each point in this direction, i.e, $r0, r0+1, \ldots, r1$ totaling the
        number of elements + 1.

    \end{itemize}

\end{description}

\subsubsection{Description for Multiple Segmented Geometries}

This feature allows users to enter a complex sequence of segments for each of
the $(x,y,z)$ directions. Each segment set is defined in $(x,y,z)$ sections.
Lines 5-8 describe the $x$-dimension, lines 9-12 the $y$-dimension,
and lines 13-16 the $z$-direction.

\begin{description}

  \item[Line 5] The number of segments, $nsegs$, in the $x$-direction

  \item[Line 6] The number of elements in each segment. There should be $nsegs$
    numbers, $(nelx\_1, nelx\_2, \ldots)$

  \item[Line 7] The start (and end) coordinates for each segment in this
    direction.  There should be $nsegs+1$ numbers, $(x_0, x_1, \ldots,
    x_{nsegs})$

  \item[Line 8] The distribution of each segment. Uniform spacing corresponds
    to 1; otherwise a geometric sequence is generated.  In conclusion, a
    segment between $x_{i-1}$ and $x_i$ is filled with $nelx_i$ elements
    determined by the geometric ratio given for that segment.

  \item[Lines 9-12] Description of segments for $y$-dimension.  Follows the
    same format as lines 5-8.

  \item[Lines 13-16 (in 3D)] Description of segments for $z$-dimension.
    Follows the same format as lines 5-8.

\end{description}

\subsubsection{Footer for all geometries}

The last lines of the .box file describe the boundary conditions.  There is one
line of boundary conditions for each fld field indicated (including MHD).  The
order of the boundary conditions are: west, east, south, north, bottom, top.
Genbox expects each boundary condition to be 3 characters followed by a comma,
so it is important maintain the proper spacing.  See Section~\ref{sec:boundary}
for available options \textbf{Note: there should be no blank lines at the end
  of the .box file.  Extra whitespace will cause genbox to search for another
new tensor box data.}


% START WIP

% END WIP

\subsection{Example .box files}

\subsubsection{Uniformly distributed mesh}

Suppose you wish to simulate flow through an axisymmetric pipe, of radius
\(R=0.5\) and length \(L=4\).  You estimate that you will need 3 elements in
radial (\(y\)) direction, and 5 in the \(x\) direction, as depicted in Fig.
\ref{fig:mesh_axi1}.  This would be specified by the following input file
(called {\em pipe.box}) to genbox:

\begin{verbatim}
axisymmetric.rea
2                      spatial dimension
1                      number of fields
#
#    comments:   This is the box immediately behind the 
#                refined cylinder in Ugo's cyl+b.l. run.
#
#
#========================================================
#
Box 1                         Pipe
-5 -3                         Nelx  Nely
0.0   4.0   1.0               x0  x1   ratio
0.0   0.5   1.0               y0  y1   ratio
v  ,O  ,A  ,W  ,   ,          BC's:  (cbx0, cbx1, cby0, cby1, cbz0, cbz1)
\end{verbatim}
\begin{figure}
  \centering
  \includegraphics[width=0.8\textwidth]{Figs/mesh_axi1}
  \caption{Axisymmteric pipe mesh}
  \label{fig:mesh_axi1}
\end{figure}
\noindent
\begin{itemize}
  \item
    The first line of this file supplies the name of an existing 2D .rea file
    that has the appropriate run parameters (viscosity, timestep size, etc.).
    These parameters can be modified later, but it is important that 
    axisymmetric.rea be a 2D file, and not a 3D file.
  \item
    The second line indicates the number of fields for this simulation, in
    this case, just 1, corresponding to the velocity field (i.e., no heat 
    transfer).
  \item
    The next set of lines just shows how one can place comments into a genbox
    input file.
  \item
    The line that starts with ``Box'' indicates that a new box is starting,
    and that the following lines describe a typical box input.  Other possible
    key characters (the first character of Box, ``B'') are ``C'' and ``M'',
    more on those later.
  \item
    The first line after ``Box'' specifies the number of elements in the
    \(x\) and \(y\) directions.   The fact that these values are negative indicates
    that you want genbox to automatically generate the element distribution 
    along each axis, rather than providing it by hand.  (More on this below.)
  \item
    The next line specifies the distribution of the 5 elements in the \(x\) direction.
    The mesh starts at \(x=0\) and ends at \(x=4.0\).  The {\em ratio} indicates the
    relative size of each element, progressing from left to right.  Here, 
  \item
    The next line specifies the distribution of the 3 elements in the \(y\) direction,
    starting at \(y=0\) and going to \(y=0.5\).  Again, 
    {\em ratio}=1.0 indicates that the elements will be of uniform height.
  \item
    The last line specifies boundary conditions on each of the 4 sides of the
    box:  
    \begin{itemize}
      \item
        Lower-case {\em v} indicates that the left (\(x\)) boundary is to be a velocity
        boundary condition, with a user-specified distribution determined by 
        routine {\em userbc} in the .usr file.  (Upper-case \(V\) would indicate that
        the velocity is constant, with values specified in the .rea file.)
      \item
        {\em O} indicates that the right (\(x\)) boundary is an outflow boundary -- the
        flow leaves the domain at the left and the default exit pressure is \(p=0\).
      \item
        {\em A} indicates that the lower (\(y\)) boundary is the axis---this condition
        is mandatory for the axisymmetric case, given the fact that the lower domain
        boundary is at \(y=0\), which corresponds to \(r=0\).
      \item
        {\em W} indicates that the upper (\(y\)) boundary is a wall.  This would be
        equivalent to a {\em v} or {\em V} boundary condition, with \(\bu=0\).
    \end{itemize}
\end{itemize}


\subsubsection{Graded Mesh}
\begin{figure}
  \centering
  \includegraphics[width=0.8\textwidth]{Figs/mesh_axi2}
  \caption{Axisymmteric pipe mesh, graded}
  \label{fig:mesh_axi2}
\end{figure}

Suppose you wish to have the mesh be graded,
that you have increased resolution near the wall.
In this case you change {\em ratio} in the \(y\)-specification
of the element distribution.  For example, changing the 3 lines
in the above genbox input file from

\begin{verbatim}
-5 -3                         Nelx  Nely
0.0   4.0   1.0               x0  x1   ratio
0.0   0.5   1.0               y0  y1   ratio
\end{verbatim}

\noindent
to

\begin{verbatim}
-5 -4                         Nelx  Nely
0.0   4.0   1.0               x0  x1   ratio
0.0   0.5   0.7               y0  y1   ratio
\end{verbatim}

\noindent
yields the mesh shown in Fig. \ref{fig:mesh_axi2}.


\subsubsection{User-Specified Distribution}
\begin{figure}
  \centering
  \includegraphics[width=0.6\textwidth]{Figs/mesh_axi3}
  \caption{Axisymmteric pipe mesh, user specified}
  \label{fig:mesh_axi3}
\end{figure}

You can also specify your own, precise, distribution of element
locations.   For example, another graded mesh similar to the
one of the preceding example could be built by changing the
genbox input file to contain:


\begin{verbatim}
-5  4                                               Nelx  Nely
0.0   4.0   1.0                                     x0  x1   ratio
0.000    0.250    0.375    0.450    0.500           y0  y1 ... y4
\end{verbatim}

\noindent
Here, the positive number of elements for the \(y\) direction indicates
that genbox is expecting {\tt Nely+1} values of \(y\) positions on the
\(y\)-element distribution line.   This is the genbox default, which
explains why it corresponds to {\tt Nely} \(>\) 0.  The corresponding mesh
is shown in Fig. \ref{fig:mesh_axi3}.

\subsubsection{ Annulus Geometry }
\begin{verbatim}base.rea
2                         spatial dimension
2                         number of fields
Y1
-4  -8                     nelx,nely
0 0                        x0 y0 center
ccobc                      nelx+1 letter defining each edge shape
1 5 1.5                    x0 x1 ratio
0 1 1                      y0 y1 ratio
SYM,SYM,   ,               V bc's ~! NB:  3 characters each~! 
f  ,f  ,   ,               T bc's ~!      You must have 2 spaces!!
\end{verbatim}

\subsubsection{ MHD}
\begin{verbatim}base.rea
3                          spatial dimensions: <0 for .re2
2.1                        number of fields       ~: v,T + B
Box 1
-4  -4  -8                 (nelx,nely,nelz for Box)
0.0 1.0 1.0                (x0, x1 ratio or xe_i)
0.0 1.0 1.0                (y0, y1 ratio or ye_j)
0.0 1.0 1.0                (z0, z1 ratio or ze_k)
P  ,P  ,P  ,P  ,P  ,P      (cbx0,  cbx1,  cby0,  cby1,  cbz0,  cbz1)  Velocity (3 characters)
P  ,P  ,P  ,P  ,P  ,P      (cbx0,  cbx1,  cby0,  cby1,  cbz0,  cbz1)  Temperature
P  ,P  ,P  ,P  ,P  ,P      (cbx0,  cbx1,  cby0,  cby1,  cbz0,  cbz1)  Magnetic Field
\end{verbatim}

\subsubsection{ Cicular, 1/4 , no velocity, Passive scalars}
\begin{verbatim}base.rea
2                          spatial dimensions: <0 for .re2
-3                         number of fields       ~: (NO VELOCITY) heat + 2 passive scalars
ctest                      Signals circular geometry
0 0                        x0,y0 center
-2 -6                      nelx nely for box
1  2  1                    x0 x1 ratio
90 180 1                   y0 y1 ration~: degrees!!!!
SYM,SYM,   ,   ,           Temperature Boundary
E  ,E  ,   ,   ,           PS1
P  ,P  ,   ,   ,           PS2
\end{verbatim}


\section{Mesh Modification in Nek5000}

For complex shapes, it is often convenient to modify the mesh
direction in the simulation code, Nek5000.  This can be done
through the usrdat2 routine provided in the .usr file.
The routine usrdat2 is called by nek5000 immediately after
the geometry, as specified by the .rea file, is established.
Thus, one can use the existing geometry to map to a new geometry
of interest.

For example, suppose you want the above pipe geometry to have
a sinusoidal wall.  Let \(\bx := (x,y)\) denote the old geometry,
and \(\bx' := (x',y')\) denote the new geometry.  For a domain
with \(y\in [0,0.5]\), the following function will map the straight
pipe geometry to a wavy wall with amplitude \(A\), wavelength \(\lambda\):
\begin{eqnarray*}
  y'(x,y) = y  + y A \sin( 2 \pi x / \lambda ).
\end{eqnarray*}
Note that, as \(y \longrightarrow 0\), the perturbation, 
\(yA \sin( 2 \pi x / \lambda )\), goes to zero.  So, near the axis,
the mesh recovers its original form.

In nek5000, you would specify this through usrdat2 as follows


\begin{verbatim}
subroutine usrdat2
include 'SIZE'
include 'TOTAL'

real lambda

ntot = nx1*ny1*nz1*nelt

lambda = 3.
A      = 0.1

do i=1,ntot
argx         = 2*pi*xm1(i,1,1,1)/lambda
ym1(i,1,1,1) = ym1(i,1,1,1) + ym1(i,1,1,1)*A*sin(argx)
enddo

param(59) = 1.  ! Force nek5 to recognize element deformation.

return
end
\end{verbatim}
\noindent
Note that, since nek5000 is modifying the mesh, postx will not
recognize the current mesh unless you tell it to, because postx
looks to the .rea file for the mesh geometry.  The only way for
nek5000 to communicate the new mesh to postx is via the .fld
file, so you must request that the geometry be dumped to the
.fld file.   This is done by modifying the OUTPUT SPECIFICATIONS,
which are found near the bottom of the .rea file.  Specifically,
change

\begin{verbatim}
***** OUTPUT FIELD SPECIFICATION *****
6 SPECIFICATIONS FOLLOW
F      COORDINATES
T      VELOCITY
T      PRESSURE
T      TEMPERATURE
F      TEMPERATURE GRADIENT
0      PASSIVE SCALARS
\end{verbatim} 

\noindent
to

\begin{verbatim}
***** OUTPUT FIELD SPECIFICATION *****
6 SPECIFICATIONS FOLLOW
T      COORDINATES                       <------  CHANGE HERE
T      VELOCITY
T      PRESSURE
T      TEMPERATURE
F      TEMPERATURE GRADIENT
0      PASSIVE SCALARS
\end{verbatim} 

\noindent
The result of above changes is shown in Fig. \ref{fig:wavypipe}.
\begin{figure}
  \centering
  \includegraphics[width=0.8\textwidth]{Figs/wavypipe}
  \caption{Axisymmteric pipe mesh}
  \label{fig:wavypipe}
\end{figure}

\subsection{Cylindrical/Cartesian-transition Annuli}
\begin{figure}
  \centering
  \subfloat[Annuli mesh]{\includegraphics[width=0.3\textwidth]{Figs/cylbox_2d}\label{fig:cylbox_2d}}
  \quad\quad\quad
  \subfloat[Annuli mesh] {\includegraphics[width=0.3\textwidth]{Figs/cylbox_2d}\label{fig:cylbox_2da}} 
  \caption{Cylinder mesh}
\end{figure}


An updated version of genb6, known as genb7, is currently under development
and designed to simply/automate the construction of cylindrical annuli, 
including {\em basic} transition-to-Cartesian elements.   More sophisticated
transition treatments may be generated using the GLOBAL REFINE options in
prenek or through an upgrade of genb7, as demand warrants.
Example 2D and 3D input files are provided in the nek5000/doc files
{\em box7.2d} and {\em box7.3d}.
Figure \ref{fig:cylbox_2d} shows a 2D example generated using 
the {\em box7.2d} input file, which reads:
\begin{verbatim}
x2d.rea
2                      spatial dimension
1                      number of fields
#
#    comments
#
#
#========================================================
#
Y                   cYlinder
3 -24 1             nelr,nel_theta,nelz
.5 .3               x0,y0 - center of cylinder
ccbb                descriptors: c-cyl, o-oct, b-box (1 character + space)
.5 .55 .7 .8        r0 r1 ... r_nelr
0  1  1             theta0/2pi theta1/2pi  ratio 
v  ,W  ,E  ,E  ,    bc's (3 characters + comma)
\end{verbatim}

\noindent
An example of a mesh is shown in Fig. \ref{fig:cylbox_2d}.   The mesh has been quad-refined
once with oct-refine option of prenek. The 3D counterpart to this 
mesh could joined to a hemisphere/Cartesian transition built with
the spherical mesh option in prenek. 

%\input{mesh101}
%\section{Curvilinear geometry}
\section{Boundary and Initial Conditions}

\subsection{Boundary Conditions}\label{sec:boundary}

The boundary conditions for the governing equations
given in the previous section are now described.
%Note that if the boundary conditions (for any field variable)
%are nonzero, the inhomogeneities can either be defined as constant, or as fortran functions of
%appropriate parameters such as space, time, temperature, etc.
%In this case, the user is responsible for using relevant variables in all
%fortran function definitions.

The boundary conditions can be imposed in various ways:
\begin{itemize}
\item when the mesh is generated with \texttt{genbox}, as will be explained in Section~\ref{sec:genbox}
\item when the .rea file is read in PRENEK or directly in the .rea file
\item directly in the .rea file
\item in the subroutine \texttt{userbc} 
\end{itemize} 

The general convention for boundary conditions in the .rea file is 
\begin{itemize}
\item upper case letters correspond to Primitive boundary conditions, as given in Table~\ref{tab:primitiveBCf, tab:primitiveBCt}
\item lower case letters correspond to user defined boundary conditions, see Table~\ref{tab:userBCf,tab:userBCt}
\end{itemize}

Since there are no supporting tools that will correctly populate the .rea file with the appropriate values, temperature, velocity, and flux boundary conditions are typically lower case and values must be specified in the \texttt{userbc} subroutine in the .usr file. %In this case PARAMs in the .rea file are dummies.
\subsection{Fluid Velocity}
  
Two types of boundary conditions are applicable to the
fluid velocity : essential (Dirichlet) boundary
condition in which the velocity is specified;
natural (Neumann) boundary condition in which the traction
is specified.
For segments that constitute the boundary \(\partial \Omega_f\), see Fig.~\ref{fig:domains},
one of these two types of boundary conditions must be
assigned to each component of the fluid velocity.
The fluid boundary condition can be {\em all Dirichlet}
if all velocity components of \(\bu\) are
specified; or it can be {\em all Neumann} if all traction components
\({\bf t} = [-p {\bf I} + \mu (\nabla \bu +
(\nabla \bu)^{T})] \cdot {\bf n}\), where
\({\bf I}\) is the identity tensor, \({\bf n}\) is the unit normal
and \(\mu\) is the dynamic viscosity, are specified;
or it can be {\em mixed Dirichlet/Neumann}
if Dirichlet and Neumann conditions are selected for different
velocity components.
Examples for all Dirichlet, all Neumann and mixed Dirichhlet/Neumann
boundaries are wall, free-surface and symmetry, respectively.
If the nonstress formulation is selected, then traction
is not defined on the boundary.
In this case, any Neumann boundary condition imposed must be homogeneous;
i.e., equal to zero.
In addition, mixed Dirichlet/Neumann boundaries must be aligned with
one of the Cartesian axes.

For flow geometry which consists of
a periodic repetition of a particular geometric unit,
the periodic boundary conditions can be imposed,
as illustrated in Fig.~\ref{fig:domains}.

\begin{table}
\begin{tabular}{ |l|l|l|l| }
   \hline
   Identifier & Description & Parameters&No of Parameters\\ \hline \hline
P    &   periodic                        &   periodic element and face & 2 \\
V    &   Dirichlet velocity              &   u,v,w                      &3 \\
O    &   outflow                         &   -                          &0  \\
W    &   wall (no slip)                  &   -                         & 0    \\                           
F    &   flux                            &   flux                      & 1\\
SYM  &   symmetry                        &   -                         & 0\\
A    &   axisymmetric boundary          &    -                         & 0\\
MS   &   moving boundary                 &   -                         & 0\\
ON   &   Outflow, Normal   &  -  &  0\\
E   &   Interior boundary   &  Neighbour element ID  &  2\\
   \hline
\end{tabular}
\caption{Primitive boundary conditions (flow velocity)}\label{tab:primitiveBCf}
\end{table}

\begin{table}
\begin{tabular}{ |l|l| }
   \hline
   Identifier & Description\\ \hline \hline
v  &      user defined Dirichlet velocity\\
t   &     user defined Dirichlet temperature\\
f    &    user defined flux\\
   \hline
\end{tabular}
\caption{User defined boundary conditions (flow velocity)}\label{tab:userBCf}
\end{table}
%\begin{itemize}
%\item 'MS ' (fs-hydro)
%\item 'O  ' 
%\item 'ON '    (blasius)
%\item 'S  '  (solid.rea)
%\end{itemize}

\paragraph*{}
The open(outflow) boundary condition ("O") arises as a natural boundary condition from the variational formulation of Navier Stokes. We identify two situations
\begin{itemize}
\item In the non-stress formulation, open boundary condition ('Do nothing')
\begin{equation}
[-p\vect I + \nu(\nabla \vect u)]\cdot \vect n=0
\end{equation}
\item In the stress formulation, free traction boundary condition
\begin{equation}
[-p\vect I + \nu(\nabla \vect u+\nabla \vect u^T)]\cdot \vect n=0
\end{equation}

\item the symmetric boundary condition ("SYM") is given as
\begin{eqnarray}
\vect u \cdot \vect n&=&0\ ,\\
(\nabla \vect u \cdot \vect t)\cdot \vect n&=&0
\end{eqnarray}
where \(\vect n\) is the normal vector and \(\vect t\) the tangent vector. If the normal and tangent vector are not aligned with the mesh the stress formulation has to be used.


\item the periodic boundary condition ("P") needs to be prescribed in the .rea file since it already assigns the last point to first via \(\vect u(\vect x)=\vect u(\vect x + L) \), where \(L\) is the periodic length.

\item the wall boundary condition ("W") corresponds to \(\vect u=0\).
\end{itemize}

%
%\begin{figure}
%\vspace{8.5in}
%\end{figure}
%
For a fully-developed flow in such a configuration, one can
effect great computational efficiencies by considering the
problem in a single geometric unit (here taken to be of
length L), and requiring periodicity of the field variables.
Nek5000 requires that the pairs of sides (or faces, in
the case of a three-dimensional mesh) identified as periodic
be identical (i.e., that the geometry be periodic).

For an axisymmetric flow geometry, the axis boundary
condition is provided for boundary segments that lie
entirely on the axis of symmetry.
This is essentially a symmetry (mixed Dirichlet/Neumann)
boundary condition
in which the normal velocity and the tangential traction
are set to zero.

For free-surface boundary segments, the inhomogeneous
traction boundary conditions
involve both the surface tension coefficient \(\sigma\)
and the mean curvature of the free surface.
\subsubsection{Passive scalars and Temperature}
  
The three types of boundary conditions applicable to the
temperature are: essential (Dirichlet) boundary
condition in which the temperature is specified;
natural (Neumann) boundary condition in which the heat flux
is specified; and mixed (Robin) boundary condition
in which the heat flux is dependent on the temperature
on the boundary.
For segments that constitute the boundary
\(\partial \Omega_f' \cup \partial \Omega_s'\) (refer to Fig. 2.1),
one of the above three types of boundary conditions must be
assigned to the temperature.

The two types of Robin boundary condition for temperature
are : convection boundary conditions for which the heat
flux into the domain depends on the heat transfer coefficient
\(h_{c}\) and the difference between the environmental temperature
\(T_{\infty}\) and the surface temperature; and radiation
boundary conditions for which the heat flux into the domain
depends on the Stefan-Boltzmann constant/view-factor
product \(h_{rad}\) and the difference between the fourth power
of the environmental temperature \(T_{\infty}\) and the fourth
power of the surface temperature.
\begin{table}
\begin{tabular}{ |l|l|l|l| }
   \hline
   Identifier & Description & Parameters&No of Parameters\\ \hline \hline
T    &   Dirichlet temperature/scalar    &   value                      &1 \\
O    &   outflow                         &   -                          &0  \\            
P    &   periodic boundary               &    -                         & 0\\
I    &   insulated (zero flux) for temperature&                        & 0\\
   \hline
\end{tabular}
\caption{Primitive boundary conditions (Temperature and Passive scalars)}\label{tab:primitiveBCt}
\end{table}

\begin{table}
\begin{tabular}{ |l|l| }
   \hline
   Identifier & Description\\ \hline \hline
t  &      user defined Dirichlet temperature\\
c   &     Newton cooling\\
f    &    user defined flux\\
   \hline
\end{tabular}
\caption{User defined boundary conditions (Temperature and Passive scalars)}\label{tab:userBCt}
\end{table}

\paragraph*{}
\begin{itemize}
\item open boundary condition ("O")
\begin{equation}
k(\nabla T)\cdot \vect n=0
\end{equation}
\item insulated boundary condition ("I")
\begin{equation}
k(\nabla T)\cdot \vect n=0
\end{equation}
where \(\vect n\) is the normal vector and \(\vect t\) the tangent vector. If the normal and tangent vector are not aligned with the mesh the stress formulation has to be used.

\item the periodic boundary condition ("P") needs to be prescribed in the .rea file since it already assigns the last point to first via \(\vect u(\vect x)=\vect u(\vect x + L) \), where \(L\) is the periodic length.

\item Newton cooling boundary condition ("c")
\begin{equation}
k(\nabla T)\cdot \vect n=h(T-T_{\infty})
\end{equation}
\item flux boundary condition ("f")
\begin{equation}
k(\nabla T)\cdot \vect n=f
\end{equation}
\end{itemize}

%\subsubsection*{Passive scalars}
The boundary conditions for the passive scalar fields
are analogous to those used for the temperature field.
Thus, the temperature boundary condition
menu will reappear for each passive scalar field so that the
user can specify an independent set of boundary conditions
for each passive scalar field. 

\input{internal_bc}
\input{intial_cond}

\section{Mesh Partioning for Parallel Computing with Genmap}\label{sec:genmap}
\input{genmap}

\chapter{Performing large scale simulations  in Nek5000}
\section{Large scale simulations}
\input{large_scale}
\section{Parallelism in Nek5000}
\input{parallelism}
%\end{comment}
%\begin{comment}

\chapter{Routines of interest}
The most common routines needed in nek500 are
\subsection{Naming conventions}
\begin{itemize}
\item  {\tt{subroutine f(a,b,c)}}
a- returned variable
b,c -input data
\item {\tt{op}[]} represent operations on operators
\item {\tt{c}[]}  operations on constants
\item {\tt{gl}[]} global operations
\item {\tt{col2}[]} ---
\item {\tt{col3}[]} --
\end{itemize}

\subsection{Subroutines}
{\tt subroutine rescale\_x(x,x0,x1)}

    Rescales the array x to be in the range (x0,x1). This is usually called from usrdat2 in the .usr file 
    
{\tt subroutine normvc(h1,semi,l2,linf,x1,x2,x3)}

    Computes the error norms of a vector field variable(x1,x2,x3) defined on mesh 1, the velocity mesh. The error norms are normalized with respect to the volume, with the exception on the infinity norm, linf. 
    
{\tt subroutine comp\_vort3(vort,work1,work2,u,v,w)}

    Computes the vorticity (vort) of the velocity field, (u,v,w) 
    
{\tt subroutine lambda2(l2)}

    Generates the Lambda-2 vortex criterion proposed by Jeong and Hussain (1995) 
    
{\tt subroutine planar\_average\_z(ua,u,w1,w2)}

    Computes the r-s planar average of the quantity u. 
    
{\tt subroutine torque\_calc(scale,x0,ifdout,iftout)}

    Computes torque about the point x0. Here scale is a user supplied multiplier so that the results may be scaled to any convenient non-dimensionalization. Both the drag and the torque can be printed to the screen by switching the appropriate ifdout(drag) or iftout(torque) logical. 
    
{\tt subroutine set\_obj}

    Defines objects for surface integrals by changing the value of hcode for future calculations. Typically called once within userchk (for istep = 0) and used for calculating torque. (see above) 
    
{\tt subroutine avg1(avg,f, alpha,beta,n,name,ifverbose)}

{\tt subroutine avg2(avg,f, alpha,beta,n,name,ifverbose)}

{\tt subroutine avg3(avg,f,g, alpha,beta,n,name,ifverbose)}

    These three subroutines calculate the (weighted) average of f. Depending on the value of the logical, ifverbose, the results will be printed to standard output along with name. In avg2, the f component is squared. In avg3, vector g also contributes to the average calculation. 
    
{\tt subroutine outpost(x,vy,vz,pr,tz,' ')}

    Dumps the current data of x,vy,vz,pr,tz to an {\tt .fld} or {\tt .f0????} file for post processing. 
    
{\tt subroutine platform\_timer(ivrb)}

    Runs the battery of timing tests for matrix-matrix products,contention-free processor-to-processor ping-pong tests, and {\tt mpi\_all\_reduce} times. Allows one to check the performance of the communication routines used on specific platforms. 
    
{\tt subroutine quickmv}

    Moves the mesh to allow user affine motion. 
    
{\tt subroutine runtimeavg(ay,y,j,istep1,ipostep,s5)}

    Computes,stores, and (for ipostep!0) prints runtime averages of j-quantity y (along w/ y itself unless ipostep<0) with j + 'rtavg\_' + (unique) s5 every ipostep for istep>=istep1. s5 is a string to append to rtavg\_ for storage file naming. 
    
{\tt subroutine lagrng(uo,y,yvec,uvec,work,n,m)}

    Compute Lagrangian interpolant for uo 
    
{\tt subroutine opcopy(a1,a2,a3,b1,b2,b3)}

    Copies b1 to a1, b2 to a2, and b3 to a3, when ndim = 3, 
    
{\tt subroutine cadd(a,const,n)}

    Adds const to vector a of size n. 
    
{\tt subroutine col2(a,b,n)}

    For n entries, calculates a=a*b. 
    
{\tt subroutine col3(a,b,c,n)}

    For n entries, calculates a=b*c. 
\subsection{Functions}

{\tt function glmax(a,n)}
{\tt function glamax(a,n)}
{\tt function iglmax(a,n)}
    Calculates the (absolute) max of a vector that is size n. Prefix i implies integer type. 
{\tt function i8glmax(a,n)}
    Calculates the max of an integer*8 vector that is size n. 
{\tt function glmin(a,n)}
{\tt function glamin(a,n)}
{\tt function iglmin(a,n)}
    Calculates the (absolute) min of a vector that is size n. Prefix i implies integer type. 


{\tt function glsc2(a,b,n)}
{\tt function glsc3(a,b,mult,n)}
{\tt function glsc23(z,y,z,n)}
    Performs the inner product in double precision. glsc3 uses a multiplier, mult and glsc23 performs x*x*y*z. 


{\tt function glsum(x,n)}
{\tt function iglsum(x,n)}
{\tt function i8glsum(x,n)}
    Computes the global sum of x, where the prefix, i specifies type integer, and i8 specifies type integer*8. 

\subsection{An example of specifying surface normals in the .usr file}
\begin{verbatim}

c-----------------------------------------------------------------------
      subroutine userbc (ix,iy,iz,iside,eg)
      include 'SIZE'
      include 'TOTAL'
      include 'NEKUSE'

      integer e,eg,f
      real snx,sny,snz   ! surface normals

      f = eface1(iside)
      e = gllel (eg)

      if (f.eq.1.or.f.eq.2) then      ! "r face"
         snx = unx(iy,iz,iside,e)                 ! Note:  iy,iz
         sny = uny(iy,iz,iside,e)
         snz = unz(iy,iz,iside,e)
      elseif (f.eq.3.or.f.eq.4)  then ! "s face"
         snx = unx(ix,iz,iside,e)                 !        ix,iz
         sny = uny(ix,iz,iside,e)
         snz = unz(ix,iz,iside,e)
      elseif (f.eq.5.or.f.eq.6)  then ! "t face"
         snx = unx(ix,iy,iside,e)                 !        ix,iy
         sny = uny(ix,iy,iside,e)
         snz = unz(ix,iy,iside,e)
      endif

      ux=0.0
      uy=0.0
      uz=0.0
      temp=0.0

      return
      end
\end{verbatim}  

This example will load a list of field files (filenames are read from a file) into the solver using the {\tt load\_fld()} function. After the data is loaded, the user is free to compute other postprocessing quantities. At the end the results are dumped onto a regular (uniform) mesh by a subsequent call to prepost().

Note: The regular grid data (field files) cannot be used as a restart file (uniform->GLL interpolation is unstable)!

\begin{verbatim}


     SUBROUTINE USERCHK
     INCLUDE 'SIZE'
     INCLUDE 'TOTAL'
     INCLUDE 'RESTART' 

     character*80 filename(9999)

     ntot   = nx1*ny1*nz1*nelv

     ifreguo = .true.   ! dump on regular (uniform) grid instead of GLL
     nrg     = 16       ! dimension of regular grid (nrg**ndim)
 
     ! read file-list
     if (nid.eq.0) then
        open(unit=199,file='file.list',form='formatted',status='old')
        read(199,*) nfiles
        read(199,'(A80)') (filename(i),i=1,nfiles)
        close(199)
     endif
     call bcast(nfiles,isize)
     call bcast(filename,nfiles*80)       

     do i = 1,nfiles
        call load\_ fld(filename(i))

        ! do something
        ! note: make sure you save the result into arrays which are
        !       dumped by prepost() e.g. T(nx1,ny1,nz1,nelt,ldimt)
        ...

        ! dump results into file
        call prepost(.true.,'his')
     enddo

     ! we're done
     call exitt

\end{verbatim}

\subsection{Spectral Interpolation Tool}

{\tt Check intpts().}
Monitor Points

Multiple monitor points can be defined in the file hpts.in to examine the field data at every timestep.

\begin{itemize}
\item setup an ASCII file called 'hpts.in' e.g: 
\begin{verbatim}
3 !number of monitoring points
1.1 -1.2 1.0
. . .
x y z
\end{verbatim}

\item depending on the number number of monitoring points you may need to increase {\tt lhis} in SIZE.
\item    add {\tt 'call hpts()'} to {\tt userchk()} 
\end{itemize}

\subsection{Grid to Grid Interpolation}

To interpolate an existing field file (e.g. base.fld) onto a new mesh do the following:
\begin{itemize}
\item set lpart in SIZE to a large value (e.g. 100'000 or larger) depending on your memory footprint
\item    compile Nek with MPIIO support
\item    set NSTEPS=0 in the .rea file (post-processing mode)
\item    run nek using the new geometry (e.g. new\_geom.f0000)
\item    run nek using the old geometry and add this code snipplet to userchk() 
\begin{verbatim}
 character*132  newfld, oldfld, newgfld
 data newfld, oldfld, newgfld /'new0.f0001','base.fld','new\_geom.f0000'/
 call g2gi(newfld, oldfld, newgfld) ! grid2grid interpolation
 call exitt()
\end{verbatim}
\end{itemize}
    

\subsection{Lagrangian Particle Tracking}

The interpolation tool can be used for Lagrangian particle tracking (the particles are the interpolation points).

Workflow: Set initial particle positions (e.g. reading a file particle.pos0) x\_part <- x\_pos0 

LOOP
\begin{itemize}
 \item    compute field quantities
\item    interpolate field quantities for all particles using intpts()
\item    dump/store particle data
\item    advect particles using particle\_advect()
\end{itemize}
END LOOP
\begin{verbatim}
    subroutine particle\_ advect(rtl,mr,npart,dt\_ p)
c     
c     Advance particle position in time using 4th-order Adams-Bashford.
c     U[x\_ i(t)] for a given x\_ i(t) will be evaluated by spectral interpolation.
c     Note: The particle timestep dt\_ p has be constant!
c
     include 'SIZE'
     include 'TOTAL'
        
     real rtl(mr,1)
         
     real vell(ldim,3,lpart)  ! lagged velocities 
     save vell
        
     integer icalld
     save    icalld
     data    icalld /0/
        
     if(npart.gt.lpart) then
       write(6,*) 'ABORT: npart>lpart - increase lpart in SIZE. ',nid
       call exitt
     endif 
        
    ! compute AB coefficients (for constant timestep)
     if (icalld.eq.0) then
        call rzero(vell,3*ldim*npart) ! k = 1 
        c0 = 1.
        c1 = 0.
        c2 = 0.
        c3 = 0.                       
        icalld = 1
     elseif (icalld.eq.1) then        ! k = 2
        c0 = 1.5
        c1 = -.5
        c2 = 0.
        c3 = 0.
        icalld = 2
     elseif (icalld.eq.2) then        ! k = 3
        c0 = 23.
        c1 = -16.
        c2 = 5.
        c0 = c0/12.
        c1 = c1/12.
        c2 = c2/12.
        c3 = 0.
        icalld = 3
     else                             ! k = 4
        c0 = 55.
        c1 = -59.
        c2 = 37.
        c3 = -9.
        c0 = c0/24.
        c1 = c1/24.
        c2 = c2/24.
        c3 = c3/24.
     endif

     ! compute new position x[t(n+1)]
     do i=1,npart
        do k=1,ndim
           vv = rtl(1+2*ndim+k,i)
           rtl(1+k,i) =  rtl(1+k,i) +
    \&                    dt\_p*(
    \&                    + c0*vv
    \&                    + c1*vell(k,1,i)
    \&                    + c2*vell(k,2,i)
    \&                    + c3*vell(k,3,i)
    \&                    )
           ! store velocity history 
           vell(k,3,i) = vell(k,2,i)
           vell(k,2,i) = vell(k,1,i)
           vell(k,1,i) = vv
        enddo
     enddo

     return
     end
  \end{verbatim}  

%\section{Troubleshooting}
%\input{troubleshoot}
\begin{comment}
\chapter{Postprocessing}
\section{Visualisation}
%\section{Toolboxes}

%\chapter{Algorithms}
%\section{Flow solvers}
%\section{Filtering}
\end{comment} 
\chapter{Appendix}
\section{Appendix A. Extensive list of parameters .rea file}
\subsection{Parameters}
This section tells nek5000
\begin{itemize}
\item If the input file reflects a 2D or 3D job (it should match the \textit{ldim} parameter in the SIZE file).
\item The combination of heat transfer, Stokes, Navier-Stokes, steady or unsteady to be run.
\item The relevant physical parameters.
\item The solution algorithm within nek5000 to use.
\item The timestep size or Courant number to use, or whether to run variable DT (\(dt\)), etc.
\end{itemize}
A .rea file starts with the following three parameters:
\begin{description}
\item [NEKTON VERSION] the version of nek5000
\item [DIMENSIONAL RUN] number of spatial dimensions (NDIM=2,3 - has to match the setting in the SIZE file).
\item [PARAMETERS FOLLOW] the number of parameters which are going to be followed in the .rea file.(NPARAM)
\end{description}
The latter specifies how many lines of .rea file, starting from the next line, are the parameters and have to be read by the program.\\  

% \noindent\Large\textbf{- Available Parameters}\normalsize
\subsection{Available Parameters}
% \begin{center}
%     \begin{tabular}{ | c | c | c | p{10cm} |}
%     \hline
%       number & name & def. value & comments \\ \hline
%       \textbf{P001}   & DENSITY  &  & density for the case of constant properties (see parameter \textbf{P030})\\ \hline
%       \textbf{P002}   & VISCOS   &  & kinematic viscosity (if \(<0 \rightarrow Re\), otherwise \(1/Re\)).\\
%     \hline
%     \end{tabular}
% \end{center}
\begin{description}
\item [P001  DENSITY] density for the case of constant properties (for variable density see parameter \textbf{P030}).
\item [P002  VISCOS]  kinematic viscosity (if \(<0 \rightarrow Re\), otherwise \(1/Re\)).
\item [P003  BETAG] if \(>0\), natural convection is turned on (Boussinesq approximation). {\textcolor{red}{NOT IN USE !}}
\item [P004  GTHETA] model parameter for Boussinesq approximation (see parameter P003). {\textcolor{red}{ NOT IN USE!}}
\item [P005  PGRADX] {\textcolor{red}{ NOT IN USE!}}
\item [P006  ] {\textcolor{red}{ NOT IN USE!}}
\item [P007  RHOCP] navier5.f:      param(7) = param(1)  ! rhoCP   = rho {\textcolor{red}{ NOT IN USE!}}
\item [P008  CONDUCT] conductivity for the case of constant properties (if \(<0\), it defines the Peclet number, see parameter P030). \\
connect2.f:      if(param(8) .lt.0.0) param(8)  = -1.0/param(8)\\
navier5.f:      param(8) = param(2)  ! conduct = dyn. visc
\item [P009  ] {\textcolor{red}{ NOT IN USE!}} (passed to CPFLD(2,3)!)\\
connect2.f:      CPFLD(2,3)=PARAM(9)
\item [P010  FINTIME] if \(>0\), specifies simulation end time. Otherwise, use NSTEP (P011).\\
drive2.f:      FINTIM = PARAM(10)
\item [P011  NSTEP] number of time steps.\\
connect2.f:            param(11) = 1.0\\
drive2.f:      NSTEPS = PARAM(11)
\item [P012  DT] upper bound on time step \(dt\)   (if \(<0\), then \(dt=|P012|\) constant)\\
connect2.f:            param(12) = 1.0\\
drive2.f:      DT     = abs(PARAM(12))
\item [P013  IOCOMM] frequency of iteration histories\\
drive2.f:      IOCOMM = PARAM(13)
\item [P014  IOTIME] if \(>0\), time interval to dump the fld file. Otherwise, use IOSTEP (P015).\\
drive2.f:      TIMEIO = PARAM(14)
\item [P015  IOSTEP] dump frequency, number of time steps between dumps.\\
drive2.f:      IOSTEP = PARAM(15)\\
navier5.f:      if  (iastep.eq.0) iastep=param(15)   ! same as iostep
\item [P016  PSSOLVER] heat/passive scalar solver:
	\subitem 1: Helmholz
	\subitem 2: CVODE
	\subitem 3: CVODE with user-supplied Jacobian
	\subitem Note: a negative number will set source terms to 0.
\item [P017  AXIS]  {\textcolor{red}{ NOT IN USE!}}
\item [P018  GRID] {\textcolor{red}{ NOT IN USE!}}
\item [P019  INTYPE] {\textcolor{red}{ NOT IN USE!}}\\
 connect2.f:            param(19) = 0.0
\item [P020  NORDER]  {\textcolor{red}{ NOT IN USE!}}
\item [P021  DIVERGENCE] tolerance for the pressure solver.\\
drive2.f:      TOLPDF = abs(PARAM(21))\\
hmholtz.f:      if (name.eq.'PRES'.and.param(21).ne.0) tol=abs(param(21))
\item [P022  HELMHOLTZ] tolerance for the velocity solver.\\
drive2.f:      TOLHDF = abs(PARAM(22))\\
hmholtz.f:      if (param(22).ne.0) tol=abs(param(22))\\
hmholtz.f:         if (param(22).lt.0) tol=abs(param(22))*rbn0\\
navier4.f:      if (param(22).ne.0) tol = abs(param(22))
\item [P023  NPSCAL] number of passive scalars.\\
connect2.f:      NPSCAL=INT(PARAM(23))
\item [P024  TOLREL] relative tolerance for the passive scalar solver (CVODE).\\
drive2.f:      TOLREL = abs(PARAM(24))
\item [P025  TOLABS] absolute tolerance for the passive scalar solver (CVODE).\\
drive2.f:      TOLABS = abs(PARAM(25))
\item [P026  COURANT] maximum Courant number (number of RK4 substeps if OIFS is used).\\
drive2.f:      CTARG  = PARAM(26)
\item [P027  TORDER] temporal discretization order (2 or 3).\\
drive2.f:      NBDINP = PARAM(27)
\item [P028  NABMSH] Order of temporal integration for mesh velocity.if 1, 2, or 3 use Adams-Bashforth of corresponding order. Otherwise, extrapolation of order TORDER (P027).\\
\item [P029  MHD\_VISCOS] if \(>0 \rightarrow\) magnetic viscosity, if \(<0 \rightarrow\) magnetic Reynolds number.\\
connect2.f:      if(param(29).lt.0.0) param(29) = -1.0/param(29)\\
connect2.f:      if (param(29).ne.0.) ifmhd  = .true.\\
connect2.f:         cpfld(ifldmhd,1) = param(29)  ! magnetic viscosity
\item [P030  USERVP] if
	\subitem 0: constant properties
	\subitem 1: user-defined properties via USERVP subroutine (each scalar separately)
	\subitem 2: user-defined properties via USERVP subroutine (all scalars at once)
\item [P031  NPERT]  if \(\ne 0\), number of perturbation modes in linearized N-S.\\
connect2.f:      if (param(31).ne.0.) ifpert = .true.\\
connect2.f:      if (param(31).lt.0.) ifbase = .false.   ! don't time adv base flow\\
connect2.f:      npert = abs(param(31)) 
\item [P032  NBCRE2] if \(>0\), number of BCs in .re2 file, 0: all.\\
connect2.f:      if (param(32).gt.0) nfldt = ibc + param(32)-1
\item [P033  ] {\textcolor{red}{ NOT IN USE!}}
\item [P034  ] {\textcolor{red}{ NOT IN USE!}}
\item [P035  ] {\textcolor{red}{ NOT IN USE!}}
\item [P036  XMAGNET] {\textcolor{red}{ NOT IN USE!}}
\item [P037  NGRIDS] {\textcolor{red}{ NOT IN USE!}}
\item [P038  NORDER2] {\textcolor{red}{ NOT IN USE!}}
\item [P039  NORDER3] {\textcolor{red}{ NOT IN USE!}}
\item [P040  ] {\textcolor{red}{ NOT IN USE!}}
\item [P041  ] 1 \(\rightarrow\) multiplicative SEMG\\
hsmg.f:c     if (param(41).eq.1) ifhybrid = .true. \(\leftarrow\) {\textcolor{red}{ NOT IN USE!}}
\item [P042  ] linear solver for the pressure equation, 0 \(\rightarrow\) GMRES or 1 \(\rightarrow\) PCG
\item [P043  ] 0: additive multilevel scheme - 1: original two level scheme.\\
navier6.f:      if (lx1.eq.2) param(43)=1.\\   
navier6.f:            if (param(43).eq.0) call hsmg\_setup         
\item [P044  ] 0=E-based additive Schwarz for PnPn-2; 1=A-based.

\item [P045  ] Free-surface stability control (defaults to 1.0)\\
subs1.f:      FACTOR = PARAM(45)
\item [P046  ] if \(>0\), do not set Initial Condition (no call to subroutine SETICS).\\
drive2.f:      irst = param(46)\\
ic.f:      irst = param(46)        ! for lee's restart (rarely used)\\
subs1.f:      irst = param(46)
\item [P047  ] parameter for moving mesh (Poisson ratio for mesh elasticity solve (default 0.4)).\\
mvmesh.f:      VNU    = param(47)
\item [P048  ] {\textcolor{red}{ NOT IN USE!}}
\item [P049  ] if \(<0\), mixing length factor {\textcolor{red}{ NOT IN USE!}}.\\
drive2.f:c     IF (PARAM(49) .LE. 0.0) PARAM(49) = TLFAC\\
turb.f:      TLFAC = PARAM(49)
\item [P050  ] {\textcolor{red}{ NOT IN USE!}}
\item [P051  ] {\textcolor{red}{ NOT IN USE!}}
\item [P052  HISTEP] if \(>1\), history points dump frequency (in number of steps).\\
prepost.f:      if (param(52).ge.1) iohis=param(52)
\item [P053  ] {\textcolor{red}{ NOT IN USE!}}
\item [P054  ] direction of fixed mass flowrate (1: \(x\)-, 2: \(y\)-, 3: \(z\)-direction). If 0: \(x\)-direction.\\
drive2.f:      if (param(54).ne.0) icvflow = abs(param(54))\\
drive2.f:      if (param(54).lt.0) iavflow = 1 ! mean velocity
\item [P055  ] volumetric flow rate for periodic case;  if p54\(<0\), then p55:=mean velocity.\\
drive2.f:      flowrate = param(55)
\item [P056  ] {\textcolor{red}{ NOT IN USE!}}
\item [P057  ] {\textcolor{red}{ NOT IN USE!}}
\item [P058  ] {\textcolor{red}{ NOT IN USE!}}
\item [P059  ] if \(\neq0\), deformed elements (only relevant for FDM). !=0 \(\rightarrow\) full Jac. eval. for each el.
\item [P060  ] if \(\neq0\), initialize velocity to 1e-10 (for steady Stokes problem).
\item [P061  ] {\textcolor{red}{ NOT IN USE!}}
\item [P062  ] if \(>0\), swap bytes for output.
\item [P063  WDSIZO] real output wordsize (8: 8-byte reals, else 4-byte).\\
prepost.f:      if (param(63).gt.0) wdsizo = 8         ! 64-bit .fld file
\item [P064  ] if \(=1\), restart perturbation solution\\
pertsupport.f:      if(param(64).ne.1) then !fresh start, param(64) is restart flag
\item [P065  ] number of I\/O nodes (if \(< 0\) write in separate subdirectories).
\item [P066  ] Output format: (only postx uses .rea value; other nondefault should be set in usrdat) (if \(\geq 0\) binary else ASCII).\\
connect2.f:         param(66) = 6        ! binary is default\\
connect2.f:         param(66) = 0        ! ASCII
\item [P067  ] read format (if \(\geq 0\) binary else ASCII).
\item [P068  ] averaging frequency in avg\_all (0: every timestep).
\item [P069  ] {\textcolor{red}{ NOT IN USE!}}
\item [P070  ] {\textcolor{red}{ NOT IN USE!}}
\item [P071  ] {\textcolor{red}{ NOT IN USE!}}
\item [P072  ] {\textcolor{red}{ NOT IN USE!}}
\item [P073  ] {\textcolor{red}{ NOT IN USE!}}
\item [P074  ] if \( > 0\) print Helmholtz solver iterations.\\
hmholtz.f:         if (nid.eq.0.and.ifprint.and.param(74).ne.0) ifprinthmh=.true.
\item [P075  ] {\textcolor{red}{ NOT IN USE!}}
\item [P076  ] {\textcolor{red}{ NOT IN USE!}}
\item [P077  ] {\textcolor{red}{ NOT IN USE!}}
\item [P078  ] {\textcolor{red}{ NOT IN USE!}}
\item [P079  ] {\textcolor{red}{ NOT IN USE!}}
\item [P080  ] {\textcolor{red}{ NOT IN USE!}}
\item [P081  ] {\textcolor{red}{ NOT IN USE!}}
\item [P082  ] coarse-grid dimension (2: linear). {\textcolor{red}{ NOT IN USE!}}
\item [P083  ] {\textcolor{red}{ NOT IN USE!}}
\item [P084  ] if \(<0\), force initial time step to this value.

\item [P085  ] set \(dt\) in \textit{setdt}.\\
subs1.f:            dt=dtopf*param(85)
\item [P086  ] {\textcolor{red}{RESERVED !}} if \(\neq0\), use skew-symmetric form, else convective form.\\
drive2.f:      PARAM(86) = 0 ! No skew-symm. convection for now\\
navier1.f:      if (param(86).ne.0.0) then  ! skew-symmetric form
\item [P087  ] {\textcolor{red}{ NOT IN USE!}}
\item [P088  ] {\textcolor{red}{ NOT IN USE!}}
\item [P089  ] {\textcolor{red}{RESERVED !}}
\item [P090  ] {\textcolor{red}{ NOT IN USE!}}
\item [P091  ] {\textcolor{red}{ NOT IN USE!}}
\item [P092  ] {\textcolor{red}{ NOT IN USE!}}
\item [P093  ] number of previous solutions to use for residual projection.\\
(adjust MXPREF in SIZEu accordingly)
\item [P094  ] if \(>0\), start projecting velocity and passive scalars after P094 steps
\item [P095  ] if \(>0\), start projecting pressure after P095 steps
\item [P096  ] {\textcolor{red}{ NOT IN USE!}}
\item [P097  ] {\textcolor{red}{ NOT IN USE!}}
\item [P098  ] {\textcolor{red}{ NOT IN USE!}}
\item [P099  ] dealiasing: 
	\subitem \(<0\):  disable
	\subitem 3:  old dealiasing
	\subitem 4:  new dealiasing
\item [P100  ] {\textcolor{red}{RESERVED !}} pressure preconditioner when using CG solver (0: Jacobi, \(>0\): two-level Schwarz) {\textcolor{red}{or wiseversa?}}
\item [P101  ] number of additional modes to filter (0: only last mode)\\
navier5.f:         ncut = param(101)+1
\item [P102  ] {\textcolor{red}{ NOT IN USE!}}
\item [P103  ] filter weight for last mode (\(<0\): disabled)
\item [P107  ] if \(\neq0\), add it to h2 in sethlm
\item [P116 NELX] number of elements in x for Fast Tensor Product FTP solver (0: do not use FTP).\\
NOTE: box geometries, constant properties only!
\item [P117  NELY] number of elements in y for FTP
\item [P118  NELZ] number of elements in z for FTP
\end{description}
\bigskip


\subsection{Available Logical Switches}
This part of .rea file starts with such a line:
\begin{verbatim}
n   LOGICAL SWITCHES FOLLOW 
\end{verbatim}
where \(n\) is the number of logical switches which is set in the following lines.
\subsection{Logical switches}\label{sec:switches}
Note that by default all logical switches are set to false.
\begin{description}
 \item[IFFLOW] solve for fluid (velocity, pressure).
 \item[IFHEAT] solve for heat (temperature and/or scalars).
 \item[IFTRAN] solve transient equations (otherwise, solve the steady Stokes flow).
 \item[IFADVC] specify the fields with convection.
 \item[IFTMSH] specify the field(s) defined on T mesh  (first field is the ALE mesh).
 \item[IFAXIS] axisymmetric formulation.
 \item[IFSTRS] use stress formulation in the incompressible case.
 \item[IFLOMACH] use low Mach number compressible flow.
 \item[IFMGRID] moving grid (for free surface flow).
 \item[IFMVBD] moving boundary (for free surface flow).
 \item[IFCHAR] use characteristics for convection operator.
 \item[IFSYNC] use mpi barriers to provide better timing information.
 \item[IFUSERVP] user-defined properties (e.g., \(\mu\), \(\rho\)) varying with space and time.
\end{description}

\begin{comment}
 WRITE (6,*) 'IFTRAN   =',IFTRAN
00130          WRITE (6,*) 'IFFLOW   =',IFFLOW
00131          WRITE (6,*) 'IFHEAT   =',IFHEAT
00132          WRITE (6,*) 'IFSPLIT  =',IFSPLIT
00133          WRITE (6,*) 'IFLOMACH =',IFLOMACH
00134          WRITE (6,*) 'IFUSERVP =',IFUSERVP
00135          WRITE (6,*) 'IFUSERMV =',IFUSERMV
00136          WRITE (6,*) 'IFSTRS   =',IFSTRS
00137          WRITE (6,*) 'IFCHAR   =',IFCHAR
00138          WRITE (6,*) 'IFCYCLIC =',IFCYCLIC
00139          WRITE (6,*) 'IFAXIS   =',IFAXIS
00140          WRITE (6,*) 'IFMVBD   =',IFMVBD
00141          WRITE (6,*) 'IFMELT   =',IFMELT
00142          WRITE (6,*) 'IFMODEL  =',IFMODEL
00143          WRITE (6,*) 'IFKEPS   =',IFKEPS
00144          WRITE (6,*) 'IFMOAB   =',IFMOAB
00145          WRITE (6,*) 'IFNEKNEK =',IFNEKNEK
00146          WRITE (6,*) 'IFSYNC   =',IFSYNC
00147          WRITE (6,*) '  '
00148          WRITE (6,*) 'IFVCOR   =',IFVCOR
00149          WRITE (6,*) 'IFINTQ   =',IFINTQ
00150          WRITE (6,*) 'IFCWUZ   =',IFCWUZ
00151          WRITE (6,*) 'IFSWALL  =',IFSWALL
00152          WRITE (6,*) 'IFGEOM   =',IFGEOM
00153          WRITE (6,*) 'IFSURT   =',IFSURT
00154          WRITE (6,*) 'IFWCNO   =',IFWCNO
00155          DO 500 IFIELD=1,NFIELD
00156             WRITE (6,*) '  '
00157             WRITE (6,*) 'IFTMSH for field',IFIELD,'   = ',IFTMSH(IFIELD)
00158             WRITE (6,*) 'IFADVC for field',IFIELD,'   = ',IFADVC(IFIELD)
00159             WRITE (6,*) 'IFNONL for field',IFIELD,'   = ',IFNONL(IFIELD)

\end{comment}
\section{Appendix B. Extensive list of parameters SIZE file}
\input{size_explicit}
\section{Appendix C. The fld file format}
\input{fld_format}
%\printindex 
%

\bibliographystyle{unsrt}
\renewcommand\refname{References}


\addtolength{\baselineskip}{-.1\baselineskip}
\bibliography{emmd}
\addtolength{\baselineskip}{+.111111\baselineskip}

\end{document}                          % The required last line
