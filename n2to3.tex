\subsection{Building Extruded Meshes with n2to3}

In nek5000/tools, there is a code n2to3.f that can be compiled with your
local fortran compiler (preferably not g77).
By running this code, you can extend two dimensional domains to
three dimensional ones with a user-specified number of levels in the
z-direction.  Such a mesh can then be modified using the mesh modification
approach. Assuming you have a valid two-dimensional mesh, n2to3 is straightforward
to run.  Below is a typical session, upon typing {\tt n2to3} the user is prompted at the command line

\begin{verbatim}
 Input old (source) file name:
h2e
 Input new (output) file name:
h3e
 input number of levels: (1, 2, 3,... etc.?):
16
 input z min:
0
 input z max:
16
 input gain (0=custom,1=uniform,other=geometric spacing):
1
 This is for CEM: yes or no:
n
 Enter Z (5) boundary condition (P,v,O):
v
 Enter Z (6) boundary condition (v,O):
0
 this is cbz: v  O   <---

      320 elements written to h3e.rea
FORTRAN STOP
\end{verbatim}

In this context CEM stands for computational electromagnetics, a spin-off of the original Nek5000 code.



